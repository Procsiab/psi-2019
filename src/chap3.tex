%! TEX root = main.tex
% Capitolo 3

\chapter{Vettori Aleatori}
    \section{Vettore aleatorio}
        \begin{defn}
            Consideriamo $n$ variabili aleatorie $X_1, \ldots,\, X_n$ definite tutte sullo stesso spazio di probabilità $(\Omega,\,\mathbb{F},\,P)$; diremo che sono  \textit{indipendenti} se vale:
            \begin{equation}\label{eq:Indipendenza_variabili_aleatorie}
                P(X_1 \in B_1,\, \ldots,\, X_n \in B_n) = P(X_1 \in B_1) \cdot \ldots \cdot P(X_n \in B_n)
            ,\end{equation}
            per ogni scelta dei domini regolari $B_1,\, \ldots,\, B_n$ ottenuti al più con un numero finito e numerabile di operazioni tra intervalli.
        \end{defn}
        \begin{obsv}
            Se prendiamo dei domini regolari definiti come: \[
                \forall i \in (-\infty,\, \ldots x_i]
            ,\] l'equazione \eqref{eq:Indipendenza_variabili_aleatorie} diventa:
            \begin{equation}\label{eq:Indipendenza_v_a_ripartizione}
                P(X_1 \leq x\,\, \ldots,\, X_N \leq x_n) = P(X_1 \leq x_1) \cdot \ldots \cdot P(X_n \leq x_n)
            .\end{equation}
        \end{obsv}
        \begin{prty}
            Prese $n$ variabili aleatorie $X_1,\, \ldots,\, X_n$, esse sono indipendenti se vale  \eqref{eq:Indipendenza_v_a_ripartizione} per ogni scelta di $x_1,\, \ldots,\, x_n \in \mathbb{R}$.
        \end{prty}
        \begin{obsv}
            Consideriamo $n$ variabili aleatorie indipendenti e discrete $X_1,\, \ldots,\, X_n$ con densità rispettivamente $p_{X_1},\, \ldots,\, p_{X_n}$; allora scegliendo i domini in \eqref{eq:Indipendenza_variabili_aleatorie} come: \[
                B_1 = \{x_1\},\, \ldots,\, B_n = \{x_n\}
            ,\] otteniamo che:
            \begin{align}\label{eq:Indipendenza_v_a_densità}
                P(X\ = x_1,\, \ldots,\, X_n = x_n) = P(X_1 = x_1) \cdot \ldots \cdot P(X_n = x_n) & &
                \forall  x_i \in \mathbb{R} \forall i = 1,\, \ldots,\, n
            .\end{align}
        \end{obsv}
        \begin{prty}
            Possiamo affermare che, prese $n$ variabili aleatorie discrete $X_1,\, \ldots,\, X_n$, esse sono indipendenti se vale \eqref{eq:Indipendenza_v_a_densità}.
        \end{prty}
    \section{Vettori aleatori}
        \begin{defn}
            Sia dato uno spazio di probabilità $(\Omega,\,\mathbb{F},\,P)$; un \textit{vettore aleatorio} $n$\nbdash dimensionale è una funzione vettoriale definita come:
            \begin{align*}
                \vec{X} \coloneqq (X_1,\, \ldots,\, X_{n}) & &
                \vec{X}\,:\, \Omega \mapsto \mathbb{R}^n
            ,\end{align*}
            tale che per ogni $i \in 1,\, \ldots,\, n$ ciascun $X_i$ sia una variabile aleatoria.
        \end{defn}
    \section{Funzione di ripartizione congiunta}
        \begin{defn}
            Sia $\vec{X} = (X_1,\, \ldots,\, X_{n})$ un vettore aleatorio $n$\nbdash dimensionale definito sullo spazio di probabilità $(\Omega,\,\mathbb{F},\,P)$; chiamiamo funzione di ripartizione di $\vec{X}$ (oppure funzione di ripartizione \textit{congiunta} di $X_1,\, \ldots,\, X_{n}$) la funzione: \[
                F_{\vec{X}} = F_{(X_1,\, \ldots,\, X_{n})} \,:\, \mathbb{R}^n \mapsto [0,\,1]
            ,\] definita $\forall x \in (x_1,\, \ldots,\, x_{n}) \in \mathbb{R}^n$ come: \[
                F_{(X_1,\, \ldots,\, X_{n})}(x_1,\, \ldots,\, x_{n}) \coloneqq P(X_1 \leq x_1,\, \ldots,\, X_n \leq x_n)
            .\]
        \end{defn}
        \begin{prty}\label{prty:Convergenza_ripartizione_congiunta}
            Sia $\vec{X} = (X_1,\, \ldots,\, X_{n})$ un vettore aleatorio che ammette funzione di ripartizione $F_{\vec{X}}$ e sia $\vec{x} = (x_1,\, \ldots,\, x_{n})$; allora vale: \[
                \forall k \in [1,\,n] \,:\, \lim_{x_k \to \infty} F_{\vec{X}}(\vec{x}) = 0
            ,\] mentre otteniamo:
            \begin{align*}
                \lim_{x_k \to \infty} F_{\vec{X}}(\vec{x}) &= 
                P(X_1 \leq x_1,\, \ldots,\, X_{k-1} \leq x_{k-1},\, X_{k+1} \leq x_{k+1},\, \ldots,\, X_n \leq x_n) \\
                                                           &= F_{(X_1,\, \ldots,\, X_{k-1},\, X_{k+1},\, \ldots,\, X_{n})}(x_1,\, \ldots,\, x_{k-1},\, x_{k+1},\, \ldots,\, x_{n})
            .\end{align*}
        \end{prty}
    \section{Indipendenza di variabili aleatorie}
        \begin{obsv}
            Consideriamo un vettore aleatorio bidimensionale $(X,\,Y)$ con funzione di ripartizione $F_{X,\,Y}$; la Proprietà~\ref{prty:Convergenza_ripartizione_congiunta} afferma che:
            \begin{gather*}
                \lim_{x \to \infty} F_{X,\,Y}(x,\,y) = P(Y \leq y) = F_Y(y); \\
                \lim_{y \to \infty} F_{X,\,Y}(x,\,y) = P(X \leq y = F_X(x)
            .\end{gather*}
            Se prendiamo in generale un vettore aleatorio $n$\nbdash dimensionale, applicando la Proprietà~\ref{prty:Convergenza_ripartizione_congiunta} iterativamente, otteniamo:
            \begin{align*}
                F_{X_i}(x) = \lim_{\substack{x_j \rightarrow +\infty \\ \forall j \neq i}} F_{\vec{X}}(x_1,\, \ldots,\, x_{i-1},\, x,\, x_{i+1},\, \ldots,\, x_{n})
            .\end{align*}
            Concludiamo che dalla funzione di ripartizione congiunta si possono calcolare le ripartizioni marginali, ma non vale il contrario.
        \end{obsv}
        \begin{defn}
            Le componenti di un vettore aleatorio $\vec{X} = (X_1,\, \ldots,\, X_{n})$ sono indipendenti se e solo se la funzione di ripartizione di $\vec{X}$ coincide col prodotto delle ripartizioni marginali: \[
            F_{\vec{X}} = F_{X_1} \cdot \ldots \cdot F_{X_n}
            .\] 
        \end{defn}
    \section{Vettori aleatori discreti}
        \begin{defn}
            Un vettore aleatorio $n$\nbdash dimensionale $X$ è \textit{discreto} se le sue componenti $X_1,\, \ldots,\, X_{n}$ sono variabili aleatorie discrete.
        \end{defn}
    \section{Densità discrete}
        \begin{defn}\label{defn:Densità_congiunta}
            Sia $\vec{X}$ un vettore aleatorio discreto su uno spazio di probabilità $(\Omega,\,\mathbb{F},\,P)$; la funzione: \[
                p_{\vec{X}} \coloneqq P(X_1 = x_1,\, \ldots,\, X_n = x_n)
            ,\] dove vale $\vec{x} = (x_1,\, \ldots,\, x_{n})$, si chiama  \textit{densità discreta} del vettore aleatorio $\vec{X}$ oppure densità congiunta di $X_1,\, \ldots,\, X_{n}$.
        \end{defn}
    \section{Distribuzione multinomiale}
        \begin{defn}\label{defn:Densità_multinomiale}
            Consideriamo una popolazione che contenga $k \geq 2$ tipi di oggetti, dove la proporzione degli oggetti di tipo  $i$\nbdash esimo sul totale è rappresentata da $p_i$ come:  \[
                \forall i \in [1,\, k] \,:\, p_i > 0 \land \sum_{i=1}^{k} p_i = 1
            ;\] inoltre $n$ oggetti di questa popolazione siano estratti a caso con reimmissione.
            Sia $X_i$ il numero di oggetti di tipo $i$\nbdash esimo estratti (con $i \in [1,\, k]$) e sia $\vec{X}$ il vettore aleatorio con componenti $X_1,\, \ldots,\, X_{k}$; allora il vettore $\vec{X}$ è discreto e tale che: \[
            X_1 + \ldots + X_{k} = n
            ,\] e la sua densità è detta \textit{multinomiale} di parametri $n$ e $p_1,\, \ldots,\, p_{k}$:
            \begin{align}\label{eq:Densità_multinomiale}
                P(X_1 = n_1,\, \ldots,\, X_{k} = n_k) &= \binom{n}{n_1,\, \ldots,\, n_{k}} \cdot (p_1^{n_1} \cdot \ldots \cdot p^{n_k}_{k}) \\
                                                      & \text{per } n_1,\, \ldots,\, n_{k} \in [0,\, n] \land n_1 + \ldots + n_k = n \nonumber
            .\end{align}
        \end{defn}
        \begin{prty}\label{prty:Densità_congiunta}
            Sia $p_{\vec{X}}$ la densità di un vettore aleatorio $n$\nbdash dimensionale $\vec{X}$ che assume valori in un insieme al più numerabile $S$ con probabilità 1; allora possiamo dire che:
            \begin{enumerate}
                \item $\forall \vec{x} \in \mathbb{R}^n \,:\, 0 \leq p_{\vec{X}} \leq 1 \land 
                    \forall \vec{x} \notin S \,:\, p_{\vec{X}}(\vec{x}) = 0$;
                \item $\sum_{\vec{x} \in S} è_{\vec{X}}(\vec{x}) = 1$;
                \item se $F_{\vec{X}}$ è la funzione di ripartizione di $\vec{X}$, allora: \[
                    \forall \vec{x} \in \mathbb{R}^n \,:\, F_{\vec{X}}(\vec{x}) = \!\!\!\! \sum_{\vec{y} \in S \,:\, \vec{y} \leq \vec{x}} \!\!\!\! p_{\vec{X}}(\vec{y})
                ;\] 
                \item se $B \subset \mathbb{R}^n$, allora: \[
                        P(\vec{X} \in B) = \!\!\! \sum_{\vec{x} \in B \cap S} \!\!\! p_{\vec{X}}(\vec{x})
                .\] 
            \end{enumerate}
        \end{prty}
        \begin{proof}
            La dimostrazione segue lo stesso svolgimento di quella della Proprietà~\ref{prty:Densità_discreta}.
        \end{proof}
        \begin{defn}\label{defn:Densità_marginali}
            Chiamiamo \textit{densità marginali} le densità delle componenti $X_i$ del vettore aleatori $\vec{X}$; se prendiamo la prima componente $X_1$, ricordando che:
            \begin{align*}
                p_{X_1}(x_1) &= P(X_1 = x_1) = P(X_1 = x_1,\, X_2 \in \mathbb{R},\, \ldots,\, X_n \in \mathbb{R}) \\
                             &= P(\vec{X} \in B)
            ,\end{align*}
            dove $B \coloneqq \{x_1\} \times \mathbb{R}^{n-1}$; quindi possiamo scrivere la densità come:
            \begin{equation}\label{eq:Densità_marginali}
                p_{X_1}(x_1) = \!\! \sum_{\vec{x} \in B \cap S} \!\! p_{\vec{X}}(\vec{x}) 
                = \!\! \sum_{x_2,\, \ldots,\, x_{n}} \!\! p_{\vec{X}}(x_1,\, x_2,\, \ldots,\, x_{n})
            .\end{equation}
        \end{defn}
        \begin{prty}\label{prty:Indipendenza_componenti_densità}
            Le componenti di un vettore aleatorio discreto $\vec{X} = (X_1,\, \ldots,\, X_{n})$ sono indipendenti se e solo se la densità congiunta di $\vec{X}$ coincide col prodotto di delle densità marginali $p_{X_1},\, \ldots,\, p_{X_n}$ rispettivamente di $X_1,\, \ldots,\, X_{n}$, cioè: \[
            p_{\vec{X}} = p_{X_1} \cdot \ldots \cdot p_{X_n}
            .\]
        \end{prty}
    \section{Vettori aleatori assolutamente continui}
        \begin{defn}\label{defn:Vettori_aleatori_assolutamente_continui}
            Un vettore aleatorio $\vec{X}$ $n$\nbdash dimensionale è \textit{assolutamente continuo} se esiste una funzione $f_{\vec{X}}\,:\, \mathbb{R}^n \mapsto \mathbb{R}^+$ integrabile, tale che la funzione di ripartizione di $\vec{X}$ si possa scrivere come:
            \begin{align*}
                F_{\vec{X}}(\vec{x}) = \int_{-\infty}^{x_1} \ldots 
                \int_{-\infty}^{x_n} f_{\vec{X}}(S_1,\, \ldots,\, S_{n})\, d_{S_n} \cdot \ldots \cdot d_{S_1} 
                & & \forall \vec{x} = (x_1,\, \ldots,\, x_{n})
            .\end{align*}
            La funzione $f_{\vec{X}}$ prende il nome di \textit{densità congiunta} di $X_1,\, \ldots,\, X_{n}$ oppure densità del vettore aleatorio assolutamente continuo $\vec{X}$.
        \end{defn}
        \begin{prty}\label{prty:Vettori_aleatori_assolutamente_continui}
            Sia $f_{\vec{X}}$ la densità di un vettore aleatorio $n$\nbdash dimensionale assolutamente continuo; allora possiamo affermare che:
            \begin{enumerate}
                \item $\int_{\mathbb{R}^n} f_{\vec{X}}(x_1,\, \ldots,\, x_{n})\, dx_1 \cdot \ldots \cdot dx_n = 1$;
                \item se  $F_{\vec{X}}$ è la funzione di ripartizione di $\vec{X}$, allora: \[
                    \frac{\partial^n F_{\vec{X}}(\vec{x})}{\partial x_1 \cdot \ldots \cdot \partial x_{n}} = f_{\vec{X}}(\vec{x})
                ,\] per ogni $\vec{x} \in \mathbb{R}^n$ tale che esista la derivata parziale al primo membro;
                \item se $B \subset \mathbb{R}^n$ è un \underline{dominio regolare}, allora: \[
                        P(\vec{X} \in B) = \int_{B} f_{\vec{X}}(x_1,\, \ldots,\, x_{n})\, dx_1 \cdot \ldots \cdot dx_n
                .\] 
            \end{enumerate}
        \end{prty}
        \begin{proof}
            La dimostrazione segue lo stesso svolgimento di quella della Proprietà~\ref{prty:Variabile_aleatoria_continua}.
        \end{proof}
        \begin{prty}[Densità marginale continua]\label{prty:Densità_marginale_continua}
            Se $f_{\vec{X}}$ è la densità di un vettore aleatorio $n$\nbdash dimensionale assolutamente continuo $\vec{X} = (X_1,\, \ldots,\, X_{n})$ allora $X_i$ è una variabile aleatoria assolutamente continua e la sua densità si ottiene come: \[
                f_{X_i}(x_i) = \int_{\mathbb{R}^{n-1}} f_{\vec{X}}(s_1,\, \ldots,\, s_{i-1},\, x_i,\, s_{i+1},\, \ldots,\, s_n)\, ds_1 \cdot \ldots \cdot ds_{i-1} \cdot ds_{i+1} \ldots \cdot ds_n
            .\] 
        \end{prty}
        \begin{proof}
            Consideriamo il caso $i=1$; dobbiamo dimostrare che: \[
            F_{X_i}(x) = \int_{-\infty}^{\infty} \left\{\int_{\mathbb{R}^{n-1}} f_{\vec{X}}(s_1,\, \ldots,\, s_n) ds_2 \cdot \ldots \cdot ds_n\right\}\, ds_1
            ;\] ciò si verifica dato che, se definiamo $B \coloneqq (-\infty,\, x] \times \mathbb{R}^{n-1}$, allora per il punto $(3)$ della Proprietà~\ref{prty:Vettori_aleatori_assolutamente_continui} otteniamo:
            \begin{align*}
                &\int_{-\infty}^{\infty} \left\{\int_{\mathbb{R}^{n-1}} f_{\vec{X}}(s_1,\, \ldots,\, s_n) ds_2 \cdot \ldots \cdot ds_n\right\}\, ds_1 = P(\vec{X} \in B) \\
                &= P(X_1 \leq x_1,\, X_2 \in \mathbb{R},\, \ldots,\, X_n) = P(X_1 \leq x) = F_{X_1}(x) \qedhere
            .\end{align*}
        \end{proof}
        \begin{prty}
            Le componenti di un vettore aleatorio assolutamente continuo sono indipendenti se e solo se ammettono densità congiunta che può essere ottenuta come prodotto delle densità marginali.
        \end{prty}
    \section{Funzioni di vettori aleatori}
        
    \section{Trasformazioni affini di vettori aleatori}
    \section{Convoluzione discreta e continua}
    \section{Valore atteso per funzioni di vettori aleatori}
    \section{Proprietà del valore atteso per due vettori aleatori}
    \section{Valore atteso del prodotto di due variabili aleatorie}
    \section{Varianza della somma di due variabili aleatorie}
    \section{Covarianza}
    \section{Media e varianza della Binomiale}
    \section{Coefficiente di correlazione lineare}
    \section{Somme di variabili aleatorie indipendenti}
    \section{Distribuzione del massimo e del minimo}
    \section{Media campionaria}
    \section{Legge dei grandi numeri}
    \section{Teorema del limite centrale} 
