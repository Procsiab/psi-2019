%! TEX root = main.tex
% Capitolo 6

\chapter{Verifica d'ipotesi}
    \section{Ipotesi statistica}
        \begin{defn}
            Data una popolazione con distribuzione con distribuzione $F_\vartheta$ determinata da un o più 
            parametri incogniti $\vec{\vartheta}$, chiamiamo \emph{ipotesi statistica} un'affermazione 
            su un o più parametri della popolazione.

            Se scegliamo $\vartheta \in \Theta_0 \subset \Theta$ e indichiamo con $\Theta$ lo spazio dei 
            parametri, allora indichiamo l'\emph{ipotesi nulla} come: \[
                H_0 \,:\, \vartheta \in \Theta_0
            ;\] se $\Theta_0$ si riduce a un punto abbiamo un'ipotesi semplice, altrimenti diremo che essa 
            è composta.

            Indichiamo la negazione dell'ipotesi semplice tramite: \[
                H_1 \,:\, \vartheta \in \Theta_1
            ,\] dove abbiamo indicato il complementare dello spazio dei campioni tramite 
            $\Theta_1 \coloneqq \Theta\backslash \Theta_0$, e la chiamiamo \emph{ipotesi alternativa}.
        \end{defn}
    \section{Test}
        \begin{defn}[Verifica d'ipotesi]
            Supponiamo di avere una popolazione con distribuzione $F_\vartheta$ dipendente dal parametro 
            incognito $\vartheta$, e vogliamo verificare l'ipotesi nulla $H_0$ sulla distribuzione.
            Un \emph{test per la verifica dell'ipotesi} $H_0$ è la procedura che permette di determinare se 
            i valori di un campione aleatorio sono compatibili con l'ipotesi formulata: in caso affermativo 
            \emph{accettiamo} l'ipotesi introdotta, altrimenti la \emph{rifiutiamo}.
        \end{defn}
        \begin{note}
            Accettare un'ipotesi non significa assumere che essa sia vera, ma vuol dire che basandoci sui 
            dati a disposizione non possiamo escluderla.
        \end{note}
        \begin{defn}[Regione critica]
            Sia dato un campione aleatorio $X_1,\, \ldots,\, X_{n} = \vec{X}$ estratto da una popolazione sulla 
            quale formuliamo un test con ipotesi nulla $H_0$; chiamiamo \emph{regione critica} e la indichiamo 
            tramite $C \subseteq \mathbb{R}^n$ la regione dello spazio $n$\nbdash dimensionale che determina 
            \underline{il rifiuto dell'ipotesi nulla}, in presenza della realizzazione $\vec{X} = \vec{x}$:
            \begin{align*}
                \text{accetto } H_0 & \;\;\; \text{se $\vec{x} \notin C$;} \\
                \text{rifiuto } H_0 & \;\;\; \text{se $\vec{x} \in C$.}
            \end{align*}
        \end{defn}
        \begin{obsv}
            Se vogliamo costruire la regione critica del test per verificare una ipotesi nulla 
            $H_0 \,:\, \vartheta \in \Psi$, allora dovremo individuare uno stimatore puntuale per 
            $\vartheta$, del tipo $d_n(X_1,\, \ldots,\, X_{n})$.

            Quindi, rifiuteremo l'ipotesi nulla quando $d_n(\vec{X})$ è ``lontano'' dall'insieme $\Psi$ dei 
            possibili valori del parametro $\vartheta$.

            La lontananza dello stimatore dall'insieme dei valori per il parametro è determinata dalla
            regione critica del test; quest'ultima può essere ottenuta insieme al livello di significatività 
            $\alpha$ conoscendo la distribuzione dello stimatore $d_n(\vec{X})$ quando $H_0$ è vera.
        \end{obsv}
        \begin{defn}[Errore di test]
            Consideriamo un test d'ipotesi con ipotesi nulla $H_0$; in tal caso possiamo avere solo due tipi 
            di errore:
            \begin{itemize}
                \item commettiamo errore di $\mathbf{I}$ specie quando i dati ci portano a rifiutare $H_0$ 
                    anche se in realtà essa è vera;
                \item commettiamo errore di $\mathbf{II}$ specie quando i dati ci portano ad accettare $H_0$ 
                    anche se in realtà essa è falsa.
            \end{itemize}
            Dato che la verifica di ipotesi mostra la compatibilità dell'ipotesi con i dati a disposizione, 
            l'errore di $\mathbf{I}$ specie è più grave di quello di $\mathbf{II}$ specie.
        \end{defn}
        \subsection{Livello di significatività}
            \begin{defn}
                Consideriamo un test con ipotesi nulla $H_0$; chiamiamo \emph{livello di significatività} la 
                soglia per la probabilità di errore di $\mathbf{I}$ specie, che indichiamo con $\alpha$: \[
                    P(\mathbf{I}\text{ specie}) 
                    = P_\vartheta\big((X_1,\, \ldots,\, X_{n}) \in C\big) \leq \alpha
                .\] La relazione precedente vale per ogni $\vartheta$ che soddisfa 
                $H_0\,:\, \vartheta \in \Theta_0$, inoltre possiamo chiamarla test di regione critica $C$ 
                di livello di significatività $\alpha$.
            \end{defn}
        \subsection{Funzione di potenza}
            \begin{defn}
                Chiamiamo \emph{funzione di potenza} per un test di regione critica $C$ la funzione del 
                parametro incognito $\vartheta$ definita come: \[
                    \pi(\vartheta) \coloneqq P_\vartheta(\vec{X} \in C)
                .\] Se $\vartheta \in \Theta_1$ questa funzione rappresenta la probabilità di prendere la 
                decisione giusta sull'ipotesi; altrimenti, per $\vartheta \in \Theta_0$ la funzione di 
                potenza rappresenta la probabilità di errore di $\mathbf{I}$ specie.
            \end{defn}
            \begin{obsv}[Massima potenza]
                Per costruire un test di livello di significatività $\alpha$ con la potenza massima, data la 
                statistica test $D_n$, imponiamo: \[
                    \alpha = P_{\vartheta_0}(D_n > t)
                ;\] indichiamo la soluzione come $d_{n,\,\alpha}|_{\vartheta=\vartheta_0}$, il quantile di 
                ordine $\alpha$ della distribuzione della statistica test $D_n$ quando $\vartheta=\vartheta_0$.
            \end{obsv}
            \begin{defn}[Curva operativa]
                Chiamiamo \emph{curva operativa} il grafico della seguente funzione, che indica la 
                probabilità di errore di $\mathbf{II}$ specie per $\vartheta \in \Theta_1$: \[
                    \beta(\vartheta) \coloneqq 1 - \pi(\vartheta)
                .\] 
            \end{defn}
            \begin{obsv}
                Per valutare la probabilità di errore di $\mathbf{II}$ specie, dobbiamo calcolare la potenza 
                come funzione del parametro incognito: \[
                    \pi(\vartheta) = P_\vartheta(D_n > d_{n,\,\alpha}|_{\vartheta=\vartheta_0})
                .\]
            \end{obsv}
        \subsection{Ampiezza}
            \begin{defn}
                Chiamiamo \emph{ampiezza} del test o della regione critica l'estremo superiore delle 
                probabilità d'errore di $\mathbf{I}$ specie, e la indichiamo con: \[
                    \alpha_* \coloneqq \sup_{\substack{\vartheta \in \Theta_0}} \pi(\vartheta) 
                    = \sup_{\substack{\vartheta \in \Theta_0}} P_\vartheta(\vec{X} \in C)
                .\] Diremo quindi che un test è di livello di significatività $\alpha$ solo se la sua ampiezza 
                è minore o uguale alla significatività ($\alpha_* \leq \alpha$).
            \end{defn}
        \subsection{Statistica test}
        Consideriamo la statistica $D_n = d-n(X_1,\, \ldots,\, X_{n})$, e definiamo l'intervallo 
        $C(t) = \{\vec{x} \in \mathbb{R}_n \,:\, d_n(\vec{x}) > t\}$; al variare di $t$ otteniamo una famiglia 
        di test statistici, definiti dalla regione critica $C(t)$.

        Allora chiamiamo \emph{statistica test} la statistica $D_n$, e per ogni $t = \bar{t}$ fissiamo 
        un test di regione critica $C(\bar{t})$.
        \subsection{\emph{p}\nbdash value}
            \begin{defn}
                Considerando una famiglia di test al variare del livello di significatività $\alpha$, 
                chiamiamo \emph{p\nbdash value} l'estremo inferiore della significatività per il quale 
                bisognerebbe rifiutare l'ipotesi nulla coi dati a disposizione.
            \end{defn}
            \begin{obsv}
                Sia $\alpha(t)$ l'ampiezza di un test rispetto al parametro $t$ e $d_* = d_n(\vec{x})$ il 
                valore della statistica test $D_n$ nell'esperimento considerato; allora $\alpha(t)$ è 
                una funzione decrescente di $t$, perciò avremo la minore ampiezza compatibile col rifiuto 
                dell'ipotesi nulla per $t = d_*$, quindi il \emph{p}\nbdash value sarà dato da: \[
                    \alpha(d_*) = P_{\vartheta_0}(D_n > d_*)
                .\] 
            \end{obsv}
    \section{Test per media di popolazione normale}
        \subsection{\emph{Z}\nbdash test}
            \begin{defn}[Bilatero]
                Sia $X_1,\, \ldots,\, X_{n}$ un campione aleatorio da una popolazione normale con media $\mu$ 
                incognita e varianza $\sigma^2$ nota; fissata una costante $\mu_0$, vogliamo verificare 
                l'ipotesi nulla seguente contro l'alternativa: \[
                \begin{cases}
                    H_0 : & \mu = \mu_0; \\
                    H_1 : & \mu \neq \mu_0.
                \end{cases}
                \] Dato che lo stimatore puntuale per $\mu$ è la media campionaria $\overline{X}_n$, costruiamo 
                la regione critica del test nel modo seguente: \[
                    C \coloneqq \big\{(X_1,\, \ldots,\, X_{n}) \,:\, |\overline{X}_n -\mu_0| > c\big\}
                ,\] per una opportuna costante $c$.

                Impostiamo il livello di significatività $\alpha$ per il test individuando il valore di $c$ 
                che rende pari ad $\alpha$ la probabilità di errore di $\mathbf{I}$ specie: \[
                    \alpha = P_{\mu_0}(|\overline{X}_n -\mu_0| > c)
                .\] Indicando con $Z\sim \mathcal{N}(0,\,1)$ una variabile aleatoria normale standard, sapendo 
                che $\mu = \mu_0 \implies \overline{X}_n \sim \mathcal{N}(\mu_0,\,\sigma^2 /n)$, scriviamo 
                la seguente statistica test: \[
                    \frac{\overline{X}_n -\mu_0}{\sigma /\sqrt{n}} \overset{\mu_0}{\sim} Z
                ;\] usandola possiamo riscrivere la precedente equazione come:
                \begin{align*}
                    \alpha &= P_{\mu_0}\left(\left|\frac{\overline{X}_n -\mu_0}{\sigma /\sqrt{n}}\right| 
                    > \frac{\sqrt{n}}{\sigma}\cdot c\right)
                    = P\left(|Z| > \frac{\sqrt{n}}{\sigma}\cdot c\right) \\
                    &= 2\,P\left(Z > \frac{\sqrt{n}}{\sigma}\cdot c\right)
                .\end{align*}
                Dall'ultima relazione appena trovata segue che: \[
                    \frac{\alpha}{2} = P\left(Z > \frac{\sqrt{n}}{\sigma}\cdot c\right)
                    \iff c \cdot\frac{\sqrt{n}}{\sigma} = z_{\alpha /2} \implies 
                    c = z_{\alpha /2}\cdot \frac{\sigma}{\sqrt{n}}
                .\] Sostituendo $c$ nella regione critica $C$ definita per questo test, otteniamo le condizioni 
                di rifiuto e accettazione: \[
                    \begin{cases}
                        \text{rifiutiamo } H_0 & 
                        \text{se $\left|\frac{\overline{X}_n -\mu_0}{\sigma /\sqrt{n}}\right| > z_{\alpha /2}$;} \\
                        \text{accettiamo } H_0 & 
                        \text{se $\left|\frac{\overline{X}_n -\mu_0}{\sigma /\sqrt{n}}\right| \leq z_{\alpha /2}$.}
                    \end{cases}
                \]
            \end{defn}
            \begin{defn}[Unilaterale]
                Sia $X_1,\, \ldots,\, X_{n}$ un campione aleatorio da una popolazione normale con media $\mu$ 
                incognita e varianza $\sigma^2$ nota; fissata una costante $\mu_0$, vogliamo verificare 
                l’ipotesi nulla seguente contro l’alternativa: \[
                \begin{cases}
                    H_0 : & \mu = \mu_0; \\
                    H_1 : & \mu \geq \mu_0.
                \end{cases}
                \] Dato che lo stimatore puntuale per $\mu$ è la media campionaria $\overline{X}_n$, costruiamo 
                la regione critica del test nel modo seguente: \[
                    C \coloneqq \big\{(X_1,\, \ldots,\, X_{n}) \,:\, \overline{X}_n -\mu_0 > c\big\}
                ,\] per una opportuna costante $c$, tale che il test abbia significatività $\alpha$:
                \begin{align*}
                    \alpha &= P_{\mu_0}\left(\frac{\overline{X}_n -\mu_0}{\sigma /\sqrt{n}} 
                    > \frac{\sqrt{n}}{\sigma}\cdot c\right)
                    = P\left(Z > \frac{\sqrt{n}}{\sigma}\cdot c\right) \\
                    &= 2\,P\left(Z > \frac{\sqrt{n}}{\sigma}\cdot c\right)
                .\end{align*}
                Come nella definizione dello \emph{Z}\nbdash test bilatero, otteniamo tale valore di $c$: \[
                    c = z_{\alpha /2} \cdot\sigma /\sqrt{n}
                .\] Osserviamo che le condizioni di accettazione e rifiuto sono analoghe a quelle del test 
                bilatero, senza valore assoluto.
            \end{defn}
            \begin{obsv}
                Nel caso di test unilaterale con ipotesi: \[
                    \begin{cases}
                        H_0 : & \mu = \mu_0; \\
                        H_1 : & \mu < \mu_0.
                    \end{cases}
                ,\] le condizioni di accettazione e rifiuto risultano analoghe a quelle del test bilatero, senza 
                valore assoluto e con le disuguaglianze invertite.
            \end{obsv}
        \subsection{\emph{t}\nbdash test}\label{sec:t_test}
            \begin{defn}[Bilatero]
                Sia $X_1,\, \ldots,\, X_{n}$ un campione estratto da una popolazione con densità 
                $\mathcal{N}(\mu,\,\sigma^2)$, di cui media e varianza incognite; fissata una costante 
                $\mu_0$, vogliamo verificare l'ipotesi nulla seguente contro l'alternativa: \[
                    \begin{cases}
                        H_0 : & \mu = \mu_0; \\
                        H_1 : & \mu \neq \mu_0.
                    \end{cases}
                .\] Al contrario dello \emph{Z}\nbdash test, in questo caso non conosciamo la varianza della 
                distribuzione, dunque adottiamo come suo stimatore la varianza campionaria $S_n^2$ e 
                rifiutiamo l'ipotesi nulla quando abbiamo: \[
                \left|\frac{\overline{X}_n -\mu}{S_n /\sqrt{n}}\right| =\; \text{troppo grande}
            ,\] con $S_n$ la deviazione standard campionaria (vedere la 
            Definizione~\ref{defn:Varianza_campionaria}); definiamo il ``troppo grande'' rispetto al livello 
            di significatività $\alpha$.

            La Proprietà~\ref{prty:Distribuzione_congiunta_statistiche_normale} ci permette di affermare che, 
            indicata con $D_n$ la statistica test e sapendo $H_0$ è vera: \[
                D_n \coloneqq \frac{\overline{X}_n -\mu_0}{S_n /\sqrt{n}} \sim t(n-1)
            .\] Imponiamo quindi che il complementare della probabilità dell'errore di $\mathbf{I}$ specie sia 
            uguale a $1-\alpha$ (accettare $H_0$ quando $\mu = \mu_0$): \[
                P_{\mu_0}\left(-c \leq \frac{\overline{X}_n -\mu_0}{S_n /\sqrt{n}} \leq c\right) = 1-\alpha
            .\] Sapendo che la densità \emph{t} è simmetrica rispetto allo zero, riscriviamo la precedente 
            equazione tramite l'inverso della probabilità:
            \begin{align*}
                \alpha &= 1-P(-c \leq D_n \leq c) = P(D_n \leq -c) + P(D_n \geq c) \\
                &= 2\, P(D_n \geq c)
            .\end{align*}
            Dall'ultima relazione appena trovata segue che: \[
                c = t_{\alpha /,\, n-1}
            .\] Usando $c$ come estremo della regione critica, otteniamo le condizioni di rifiuto e 
            accettazione: \[
                \begin{cases}
                    \text{rifiutiamo } H_0 & 
                    \text{se $\left|\frac{\overline{X}_n -\mu_0}{S_n /\sqrt{n}}\right| \geq t_{\alpha /2,\, n-1}$;} \\
                    \text{accettiamo } H_0 & 
                    \text{se $\left|\frac{\overline{X}_n -\mu_0}{S_n /\sqrt{n}}\right| < t_{\alpha /2,\, n-1}$.}
                \end{cases}
            \]
            \end{defn}
            \begin{defn}[Unilaterale]
                Sia $X_1,\, \ldots,\, X_{n}$ un campione estratto da una popolazione con densità 
                $\mathcal{N}(\mu,\,\sigma^2)$ di cui media e varianza incognite; fissata una costante $\mu_0$, 
                vogliamo verificare l'ipotesi nulla seguente contro l'alternativa: \[
                    \begin{cases}
                        H_0 : & \mu = \mu_0; \\
                        H_1 : & \mu \geq \mu_0.
                    \end{cases}
                .\] Con un test di livello di significatività $\alpha$, le condizioni di accettazione e 
                rifiuto sono analoghe a quelle del test bilatero, senza valore assoluto.
            \end{defn}
            \begin{obsv}
                Nel caso di test unilaterale con ipotesi: \[
                    \begin{cases}
                        H_0 : & \mu = \mu_0; \\
                        H_1 : & \mu < \mu_0.
                    \end{cases}
                ,\] le condizioni di accettazione e rifiuto risultano analoghe a quelle del test bilatero, senza 
                valore assoluto e con le disuguaglianze invertite.
            \end{obsv}
    \section{Test per varianza di popolazione normale}
    \begin{defn}
        Sia $X_1,\, \ldots,\, X_{n}$ un campione estratto da una popolazione con densità normale di media e 
        varianza incognite; fissata una costante $\sigma^2_0$, vogliamo verificare l'ipotesi nulla seguente 
        contro l'alternativa: \[
            \begin{cases}
                H_0 : & \mu = \mu_0; \\
                H_1 : & \mu \neq \mu_0.
            \end{cases}
        \] Quando $H_0$ è vera, possiamo usare il Teorema~\ref{thm:Distribuzione_congiunta_statistiche_normale} 
        per ottenere la statistica test seguente: \[
            D_n \coloneqq \frac{S_n^2}{\sigma_0^2}(n-1) \sim  \chi^2(n-1)
        ;\] Impostiamo il test a livello di significatività $\alpha$ tramite la probabilità:
        \begin{align*}
            1-\alpha = P_{(\mu,\,\sigma^2)}\left(\chi^2_{1-\alpha /2,\,n-1} < \frac{S_n^2}{\sigma_0^2}\cdot(n-1) 
            < \chi^2_{\alpha /2,\,n-1}\right)
        .\end{align*}
        Le condizioni di rifiuto e accettazione per un test di livello di significatività $\alpha$ con la 
        statistica test $D_n$ sono ottenute come: \[
            \begin{cases}
                \text{rifiutiamo } H_0 & 
                \text{se $\frac{S_n^2}{\sigma_0^2}(n-1) \leq \chi^2_{1-\alpha /2,\,n-1} \lor 
                \frac{S_n^2}{\sigma_0^2}(n-1) \geq \chi^2_{\alpha /2,\,n-1}$;} \\
                \text{accettiamo } H_0 & 
                \text{se $\chi^2_{1-\alpha/2,\,n-1} < \frac{S_n^2}{\sigma_0^2}(n-1) < \chi^2_{\alpha/2,\,n-1}$.}
            \end{cases}
        .\] 
    \end{defn}
    \section{Confronto media tra popolazioni normali indipendenti}
        \begin{defn}[$\mu_x=\mu_y=\,?,\, \sigma^2_x=\sigma^2_{x,\,0},\, \sigma^2_y=\sigma^2_{y,\,0}$]
            \hfill \\
            Consideriamo i campioni $X_1,\, \ldots,\, X_{n}$ e $Y_1,\, \ldots,\, Y_{m}$, estratti da due 
            popolazioni normali indipendenti con medie $\mu_x,\, \mu_y$ incognite e varianze 
            $\sigma_x,\, \sigma_y$ note; impostiamo quindi il test d'ipotesi seguente: \[
            \begin{cases}
                H_0 : & \mu_x = \mu_y; \\
                H_1 : & \mu_x \neq \mu_y.
            \end{cases} \implies
            \begin{cases}
                H_0 : \mu_x -\mu_y = 0; \\
                H_1 : \mu_x -\mu_y \neq 0.
            \end{cases}
            \] Dato che $\overline{X}_n$ è uno stimatore non distorto di $\mu_x$ e $\overline{Y}_m$ è uno 
            stimatore non distorto di $\mu_y$, possiamo usare $\overline{X}_n -\overline{Y}_m$ come stimatore 
            di $\mu_x -\mu_y$, nella riscrittura delle ipotesi.

            La regione critica del test avrà la forma seguente: \[
                C \coloneqq \big\{\vec{X}_n,\, \vec{Y}_m \,:\, |\overline{X}_n - \overline{Y}_m| > c\big\}
            ,\] per una opportuna costante $c$; ricordiamo che vale: \[
                \overline{X}_n -\overline{Y}_m \sim \mathcal{N}\left(\mu_x -\mu_y,\, \frac{\sigma_x^2}{n} 
                +\frac{\sigma_y^2}{m}\right)
            .\] Allora quando $H_0$ è vera la statistica test diventa: \[
                D_{n,\,m} \coloneqq \frac{\overline{X}_n -\overline{Y}_m -(\cancel{\mu_x -\mu_y})}
                {\sqrt{\frac{\sigma_x^2}{n} +\frac{\sigma_y^2}{m}}} \sim \mathcal{N}(0,\,1)
            .\] Otteniamo la significatività $\alpha$ con la seguente probabilità di accettare $H_0$: \[
            1-\alpha = P_{(\mu_x,\,\mu_y)}\left(-z_{\alpha /2} \leq \frac{\overline{X}_n 
            -\overline{Y}_m}{\sqrt{\frac{\sigma^2_y}{n} +\frac{\sigma^2_y}{m}}} \leq z_{\alpha /2}\right)
            .\] Deduciamo che le condizioni di accettazione e rifiuto per il test di significatività $\alpha$ 
            con la statistica $D_{n,\,m}$ saranno: \[
            \begin{cases}
                \text{rifiutiamo } H_0 & 
                \text{se $\frac{|\overline{X}_n -\overline{Y}_m|}{\sqrt{\frac{\sigma_x^2}{n}} 
                +\frac{\sigma_y^2}{m}} > z_{\alpha /2}$;} \\
                \vspace{-1em} \\
                \text{accettiamo } H_0 & 
                \text{se $\frac{|\overline{X}_n -\overline{Y}_m|}{\sqrt{\frac{\sigma_x^2}{n}} 
                +\frac{\sigma_y^2}{m}} \leq z_{\alpha /2}$.}
            \end{cases}
            \]
        \end{defn}
    \begin{defn}[$\mu_x=\mu_y=\,?,\, \sigma^2_x=\sigma^2_y=\,?,\, \sigma^2_x \equiv \sigma^2_y$]
            \hfill \\
            Consideriamo i campioni $X_1,\, \ldots,\, X_{n}$ e $Y_1,\, \ldots,\, Y_{m}$, estratti da due 
            popolazioni normali indipendenti con medie $\mu_x,\, \mu_y$ incognite e varianze 
            $\sigma_x,\, \sigma_y$ incognite ma supposte uguali; impostiamo quindi il test d'ipotesi seguente: \[
            \begin{cases}
                H_0 : & \mu_x = \mu_y; \\
                H_1 : & \mu_x \neq \mu_y.
            \end{cases}
            \] Come il test precedente, vogliamo rifiutare $H_0$ quando $\overline{X}_n -\overline{Y}_m$ è 
            lontano da $0$; per ottenere i limiti della regione critica, cerchiamo la distribuzione della 
            statistica test quando $H_0$ è vera.

            Per fare ciò prima di tutto definiamo lo \emph{stimatore pooled} di $\sigma^2$, dove 
            $\sigma^2 = \sigma^2_x = \sigma^2_y$ per indicare entrambe le varianze, supposte uguali:
            \begin{equation}\label{eq:Stimatore_pooled_varianza}
                S^2_p \coloneqq \frac{(n-1)S_x^2 +(m-1)S_y^2}{n+m-2} \sim \chi^2(n+m-2)
            .\end{equation}
            Per $H_0$ vera la statistica test sarà costruita come: \[
                D_{n,\,m} \coloneqq \frac{\overline{X}_n -\overline{Y}_m}{S_p\cdot \sqrt{\frac{1}{n} +\frac{1}{m}}} 
                \sim t(n+m-2)
            .\] Otteniamo la significatività $\alpha$ con la seguente probabilità di rifiutare $H_0$: \[
                \alpha = P_{(\mu_x,\,\mu_y)}\left(\left|\frac{\overline{X}_n -\overline{Y}_m}{S_p 
                \cdot \sqrt{\frac{1}{n} +\frac{1}{m}}}\right| \geq t_{\alpha /2,\,n+m-2}\right)
            .\] Deduciamo che le condizioni di accettazione e rifiuto per il test di significatività $\alpha$ 
            con la statistica $D_{n,\,m}$ saranno: \[
            \begin{cases}
                \text{rifiutiamo } H_0 & 
                \text{se $\left|\frac{\overline{X}_n -\overline{Y}_m}{\sqrt{\frac{1}{n}} 
                +\frac{1}{m}}\right| > t_{\alpha /2,\,n+m-2}$;} \\
                \vspace{-1em} \\
                \text{accettiamo } H_0 & 
                \text{se $\left|\frac{\overline{X}_n -\overline{Y}_m}{\sqrt{\frac{1}{n}} 
                +\frac{1}{m}}\right| \leq t_{\alpha /2,\,n+m-2}$.}
            \end{cases}
            \]
        \end{defn}
        \begin{defn}[$\mu_x=\mu_y=\,?,\, \sigma^2_x=\sigma^2_y=\,?,\, \sigma^2_x \equiv \sigma^2_y$]
            \hfill \\
            Consideriamo i campioni $X_1,\, \ldots,\, X_{n}$ e $Y_1,\, \ldots,\, Y_{m}$, estratti da due 
            popolazioni normali indipendenti con medie $\mu_x,\, \mu_y$ incognite e varianze 
            $\sigma_x,\, \sigma_y$ incognite (non possiamo nemmeno assumerle uguali); impostiamo quindi il 
            test d'ipotesi seguente: \[
                \begin{cases}
                    H_0 : & \mu_x = \mu_y; \\
                    H_1 : & \mu_x \neq \mu_y.
                \end{cases}
            \] La statistica test che useremo, assumendo verificata $H_0$, è la seguente:
            \begin{align*}
                D_{n,\,m} \coloneqq \frac{\overline{X}_n -\overline{Y}_m}{\sqrt{\frac{S_n^2}{n} 
                +\frac{S_m^2}{m}}} \sim \mathcal{N}(0,\,1) & &
                \text{per } n,\,m \gg 1
            .\end{align*}
            Per $n$ e $m$ molto grandi, infatti, la sua distribuzione può essere approssimata con 
            quella normale standard, semplificando la complessità data dalla dipendenza dalle varianze 
            campionarie incognite.

            Le condizioni di accettazione e rifiuto per un test di significatività 
            \underline{approssimativamente} $\alpha$ con statistica test $D_{n,\,m}$ sono: \[
            \begin{cases}
                \text{rifiutiamo } H_0 & 
                \text{se $\frac{|\overline{X}_n -\overline{Y}_m|}{\sqrt{\frac{S_x^2}{n}} 
                +\frac{S^2}{m}} > z_{\alpha /2}$;} \\
                \vspace{-1em} \\
                \text{accettiamo } H_0 & 
                \text{se $\frac{|\overline{X}_n -\overline{Y}_m|}{\sqrt{\frac{S_x^2}{n}} 
                +\frac{S_y^2}{m}} \leq z_{\alpha /2}$.}
            \end{cases}
            \]
        \end{defn}
    \section{Confronto varianza tra popolazioni normali indipendenti}
        \begin{defn}[$\mu_x =\mu_y =\,?,\, \sigma^2_x =\sigma^2_y =\,?$]
            \hfill \\
            Siano $X_1,\, \ldots,\, X_{n}$ e $Y_1,\, \ldots,\, Y_{m}$ due campioni estratti da 
            popolazioni indipendenti con distribuzioni normali, di parametri incogniti 
            $\mu_x,\,\sigma^2_x$ e $\mu_y,\,\sigma^2_y$; impostiamo il test d'ipotesi seguente: \[
                \begin{cases}
                    H_0 : & \sigma^2_x = \sigma^2_y; \\
                    H_1 : & \sigma^2_x \neq \sigma^2_y.
                \end{cases}
            \] Usando la varianza campionaria come stimatore, dal 
            Teorema~\ref{thm:Distribuzione_congiunta_statistiche_normale} sappiamo che:
            \begin{align*}
                (n-1)\cdot \frac{S_x^2}{\sigma_x^2} \sim \chi^2(n-1) & &
                (m-1)\cdot \frac{S_y^2}{\sigma_y^2} \sim \chi^2(m-1)
            ;\end{align*}
            quando si verifica l'ipotesi $H_0$ la statistica test è la seguente: \[
                D_{n,\,m} \coloneqq \frac{S_x^2}{S_y^2} \cdot \frac{\cancel{\sigma_y^2}}{\cancel{\sigma_x^2}} 
                \sim \mathcal{F}(n-1,\,m-1)
            .\] Otteniamo la significatività $\alpha$ con la seguente probabilità di accettare $H_0$: \[
            1-\alpha = P_{(\sigma_x^2,\,\sigma_y^2)}\left(\mathcal{F}_{1-\alpha /2,\,n-1,\,m-1} \leq 
            \frac{S_x^2}{S_y^2} \leq \mathcal{F}_{\alpha /2,\,n-1,\,m-1}\right)
            .\] Le condizioni di accettazione e rifiuto per un test di significatività $\alpha$ con 
            statistica test $D_{n,\,m}$ sono: \[
            \begin{cases}
                \text{rifiutiamo } H_0 & 
                \text{se $S_x^2 /S_y^2 \leq \mathcal{F}_{1-\alpha /2,\,n-1,\,m-1} \lor
                S_x^2 /S_y^2 \geq \mathcal{F}_{\alpha /2,\,n-1,\,m-1}$;} \\
                \text{accettiamo } H_0 & 
                \text{se $\mathcal{F}_{1-\alpha /2,\,n-1,\,m-1} \leq S_x^2 /S_y^2
                \leq \mathcal{F}_{\alpha /2,\,n-1,\,m-1}$.}
            \end{cases}
            \] 
        \end{defn}
    \section{Test per campioni di coppie di dati normali}
        \begin{defn}
            Consideriamo il caso di un campione costituito da $n$ coppie \\
            $(X_1,\,Y_1),\, \ldots,\, (X_n,\,Y_n)$ nel quale ciascuna coppia $(X_i,\,Y_i)$ è indipendente, 
            ma le componenti $X_i,\,Y_i$ non lo sono.

            Consideriamo il campione aleatorio $\vec{W} \coloneqq W_1 = Y_1-X_1,\, \ldots,\, W_n=Y_n-X_n$ e 
            sapendo che il valore atteso di $W$ (che indicheremo con $\mu_w$) è nullo quando si verifica 
            un evento $H_0$, possiamo costruire un test come: \[
                \begin{cases}
                    H_0 : & \mu_w = 0; \\
                    H_1 : & \mu_w \neq 0.
                \end{cases}
            \] Possiamo usare un \emph{t}\nbdash test poiché testiamo la media di una popolazione normale con 
            media e varianza incognite (Sezione~\ref{sec:t_test}); ricaviamo le seguenti regole di rifiuto e 
            accettazione con significatività $\alpha$: \[
                \begin{cases}
                    \text{rifiutiamo } H_0 & 
                    \text{se $\frac{\overline{W}_n}{S_w^2 /\sqrt{n}} \leq -t_{\alpha,\, n-1}$;} \\
                    \text{accettiamo } H_0 & 
                    \text{se $\frac{\overline{W}_n}{S_w^2 /\sqrt{n}} > -t_{\alpha,\, n-1}$.}
                \end{cases}
            .\] 
        \end{defn}
    \section{Test approssimato su proporzione}
        %TODO: Completare sezione
