%! TEX root = main.tex
% Capitolo 6

\chapter{Verifica d'ipotesi}
    \section{Ipotesi statistica}
        \begin{defn}
            Data una popolazione con distribuzione con distribuzione $F_\vartheta$ determinata da un o più 
            parametri incogniti $\vec{\vartheta}$, chiamiamo \emph{ipotesi statistica} un'affermazione 
            su un o più parametri della popolazione.

            Se scegliamo $\vartheta \in \Theta_0 \subset \Theta$ e indichiamo con $\Theta$ lo spazio dei 
            parametri, allora indichiamo l'\emph{ipotesi nulla} come: \[
                H_0 \,:\, \vartheta \in \Theta_0
            ;\] se $\Theta_0$ si riduce a un punto abbiamo un'ipotesi semplice, altrimenti diremo che essa 
            è composta.

            Indichiamo la negazione dell'ipotesi semplice tramite: \[
                H_1 \,:\, \vartheta \in \Theta_1
            ,\] dove abbiamo indicato il complementare dello spazio dei campioni tramite 
            $\Theta_1 \coloneqq \Theta\backslash \Theta_0$, e la chiamiamo \emph{ipotesi alternativa}.
        \end{defn}
    \section{Test}
        \begin{defn}[Verifica d'ipotesi]
            Supponiamo di avere una popolazione con distribuzione $F_\vartheta$ dipendente dal parametro 
            incognito $\vartheta$, e vogliamo verificare l'ipotesi nulla $H_0$ sulla distribuzione.
            Un \emph{test per la verifica dell'ipotesi} $H_0$ è la procedura che permette di determinare se 
            i valori di un campione aleatorio sono compatibili con l'ipotesi formulata: in caso affermativo 
            \emph{accettiamo} l'ipotesi introdotta, altrimenti la \emph{rifiutiamo}.
        \end{defn}
        \begin{note}
            Accettare un'ipotesi non significa assumere che essa sia vera, ma vuol dire che basandoci sui 
            dati a disposizione non possiamo escluderla.
        \end{note}
        \begin{defn}[Regione critica]
            Sia dato un campione aleatorio $X_1,\, \ldots,\, X_{n} = \vec{X}$ estratto da una popolazione sulla 
            quale formuliamo un test con ipotesi nulla $H_0$; chiamiamo \emph{regione critica} e la indichiamo 
            tramite $C \subseteq \mathbb{R}^n$ la regione dello spazio $n$\nbdash dimensionale che determina 
            \underline{il rifiuto dell'ipotesi nulla}, in presenza della realizzazione $\vec{X} = \vec{x}$:
            \begin{align*}
                \text{accetto } H_0 & \;\;\; \text{se $\vec{x} \notin C$;} \\
                \text{rifiuto } H_0 & \;\;\; \text{se $\vec{x} \in C$.}
            \end{align*}
        \end{defn}
        \begin{obsv}
            Se vogliamo costruire la regione critica del test per verificare una ipotesi nulla 
            $H_0 \,:\, \vartheta \in \Psi$, allora dovremo individuare uno stimatore puntuale per 
            $\vartheta$, del tipo $d_n(X_1,\, \ldots,\, X_{n})$.

            Quindi, rifiuteremo l'ipotesi nulla quando $d_n(\vec{X})$ è ``lontano'' dall'insieme $\Psi$ dei 
            possibili valori del parametro $\vartheta$.

            La lontananza dello stimatore dall'insieme dei valori per il parametro è determinata dalla
            regione critica del test; quest'ultima può essere ottenuta insieme al livello di significatività 
            $\alpha$ conoscendo la distribuzione dello stimatore $d_n(\vec{X})$ quando $H_0$ è vera.
        \end{obsv}
        \begin{defn}[Errore di test]
            Consideriamo un test d'ipotesi con ipotesi nulla $H_0$; in tal caso possiamo avere solo due tipi 
            di errore:
            \begin{itemize}
                \item commettiamo errore di $\mathbf{I}$ specie quando i dati ci portano a rifiutare $H_0$ 
                    anche se in realtà essa è vera;
                \item commettiamo errore di $\mathbf{II}$ specie quando i dati ci portano ad accettare $H_0$ 
                    anche se in realtà essa è falsa.
            \end{itemize}
            Dato che la verifica di ipotesi mostra la compatibilità dell'ipotesi con i dati a disposizione, 
            l'errore di $\mathbf{I}$ specie è più grave di quello di $\mathbf{II}$ specie.
        \end{defn}
        \subsection{Livello di significatività}
            \begin{defn}
                Consideriamo un test con ipotesi nulla $H_0$; chiamiamo \emph{livello di significatività} la 
                soglia per la probabilità di errore di $\mathbf{I}$ specie, che indichiamo con $\alpha$: \[
                    P(\mathbf{I}\text{ specie}) 
                    = P_\vartheta\big((X_1,\, \ldots,\, X_{n}) \in C\big) \leq \alpha
                .\] La relazione precedente vale per ogni $\vartheta$ che soddisfa 
                $H_0\,:\, \vartheta \in \Theta_0$, inoltre possiamo chiamarla test di regione critica $C$ 
                di livello di significatività $\alpha$.
            \end{defn}
        \subsection{Funzione di potenza}
            \begin{defn}
                Chiamiamo \emph{funzione di potenza} per un test di regione critica $C$ la funzione del 
                parametro incognito $\vartheta$ definita come: \[
                    \pi(\vartheta) \coloneqq P_\vartheta(\vec{X} \in C)
                .\] Se $\vartheta \in \Theta_1$ questa funzione rappresenta la probabilità di prendere la 
                decisione giusta sull'ipotesi; altrimenti, per $\vartheta \in \Theta_0$ la funzione di 
                potenza rappresenta la probabilità di errore di $\mathbf{I}$ specie.
            \end{defn}
            \begin{defn}[Curva operativa]
                Chiamiamo \emph{curva operativa} la seguente funzione, che denota la probabilità di 
                errore di $\mathbf{II}$ specie per $\vartheta \in \Theta_1$: \[
                    \beta(\vartheta) \coloneqq 1 - \pi(\vartheta)
                .\] 
            \end{defn}
        \subsection{Ampiezza}
            \begin{defn}
                Chiamiamo \emph{ampiezza} del test o della regione critica l'estremo superiore delle 
                probabilità d'errore di $\mathbf{I}$ specie, e la indichiamo con: \[
                    \alpha_* \coloneqq \sup_{\substack{\vartheta \in \Theta_0}} \pi(\vartheta) 
                    = \sup_{\substack{\vartheta \in \Theta_0}} P_\vartheta(\vec{X} \in C)
                .\] Diremo quindi che un test è di livello di significatività $\alpha$ solo se la sua ampiezza 
                è minore o uguale alla significatività ($\alpha_* \leq \alpha$).
            \end{defn}
        \subsection{Statistica test}
           %TODO: Completare sezione
        \subsection{\emph{p}\nbdash value}
            \begin{defn}
                Considerando una famiglia di test al variare del livello di significatività $\alpha$, 
                chiamiamo \emph{p\nbdash value} l'estremo inferiore della significatività per il quale 
                bisognerebbe rifiutare l'ipotesi nulla coi dati a disposizione.
            \end{defn}
    \section{Test per media di popolazione normale}
        \subsection{\emph{Z}\nbdash test}
        \subsection{\emph{t}\nbdash test}
    \section{Test approssimato su proporzione}
    \section{Confronto media tra popolazioni normali indipendenti}
    \section{Test per campione di coppie di dati normali}
    \section{Test per varianza di popolazione normale}
    \section{Confronto varianza tra popolazioni normali indipendenti}
