%! TEX root = main.tex
% Capitolo 4

\chapter{Distribuzioni notevoli}
    \section{Matrice delle covarianza}\label{defn:Matrice_covarianza}
        \begin{defn}
            Consideriamo un vettore aleatorio $n$\nbdash dimensionale $\vec{X} = (X_1,\, \ldots,\, X_{n})$ tale che sia definita la varianza di ciascuna delle sue componenti; allora chiamiamo \emph{matrice di covarianza} di $\vec{X}$ la matrice quadrata di ordine $n$ definita come:
            \begin{align*}
                \forall i,\,j \in [1,\,n] \,:\, \vec{C}_{\vec{X}} \coloneqq [c_{i,\,j}] & &
                \text{per } c_{i,\,j} = \text{Cov}(X_i,\, X_j)
            .\end{align*}
        \end{defn}
        \begin{prty}
            Sia $\vec{C}_{\vec{X}}$ la matrice di covarianza di un vettore aleatorio $\vec{X}$; allora possiamo affermare che:
            \begin{enumerate}
                \item $\vec{C}_{\vec{X}}$ è una matrice simmetrica e semi-definita positiva 
                    ($\forall \vec{X} \neq 0 \in \mathbb{R}^n \,:\, 
                    \vec{X}^{\text{\,T}}\cdot \vec{C}_{\vec{X}} \cdot \vec{X} \geq 0$);
                \item se abbiamo $\vec{A}$ una matrice $m \times n$ e $\vec{b}$ un vettore di dimensione $m$ allora la matrice di covarianza di $\vec{Y} = \vec{A} \cdot \vec{X} + \vec{b}$ si scrive come: \[
                \vec{C}_{\vec{Y}} = \vec{A} \cdot \vec{C}_{\vec{X}} \cdot \vec{A}^{\text{\,T}}
                .\] 
            \end{enumerate}
        \end{prty}
        \begin{proof}
            \hfill
            \begin{enumerate}
                \item La matrice di covarianza risulta simmetrica poiché, preso il suo elemento di posizione $i,\,j$ vale: \[
                c_{i,\,j} = \text{Cov}(X_{i},\,X_{j}) = \text{Cov}(X_{j},\,X_{i}) = c_{j,\,i}
            .\] Inoltre, se prendiamo un vettore colonna $\vec{\lambda} \in \mathbb{R}^n$ per $i\in [1,\,m]$ e $j\in [1,\,n]$ otteniamo:
                \begin{align*}
                    \vec{\lambda}^{\text{\,T}} \cdot \vec{C}_{\vec{X}} \cdot \vec{\lambda} &=
                    \sum_{i=1}^{n} \sum_{j=1}^{n}\, [\lambda_i \lambda_j \cdot c_{i,\,j}] \\
                    &= \text{E}\left(\sum_{i=1}^{n} \sum_{j=1}^{n} \lambda_i \lambda_j \cdot 
                    \left(X_i-\text{E}(X_i)\right)\left(X_j-\text{E}(X_j)\right)\right) \\
                    &= \text{E}\left(\sum_{i=1}^{n} \lambda_i \cdot \left(X_i-\text{E}(X_i)\right)^2\right) \geq 0
                .\end{align*}
            \item Se le componenti di $\vec{X}$ hanno varianza finita allora anche le componenti di $\vec{Y} = \vec{A} \cdot \vec{X} + \vec{b}$ la avranno; allora possiamo studiare la matrice di covarianza $\vec{C}_{\vec{Y}}$ considerando $\tilde{c}_{i,\,j}$ l'elemento di posto $(i,\,j)$ di $\vec{C}_{\vec{Y}}$, ottenendo:
                \begin{align*}
                    \tilde{c}_{i,\,j} = \text{Cov}\left(\vec{Y}_i,\, \vec{Y}_j\right)
                    &= \text{Cov}\left(\sum_{k=1}^{n} [a_{i,\,k} \cdot X_k + b_i],\, 
                    \sum_{\ell=1}^{n} [a_{j,\,\ell} \cdot X_\ell + b_j]\right) \\
                    &= \sum_{k=1}^{n} \sum_{\ell=1}^{n}\, \left[a_{i,\,k} \cdot a_{j,\,\ell} \cdot \text{Cov}(X_k,\, X_{\ell})\right]
                ,\end{align*}
                esso è proprio l'elemento $(i,\,j)$ della matrice $\vec{A} \cdot \vec{C}_{\vec{X}}\vec{A}^{\text{\,T}}$. \qedhere
            \end{enumerate}
        \end{proof}
    \section{Vettori gaussiani}
        \begin{defn}[Normale standard]\label{defn:Vettore_gaussiano_standard}
            Il vettore aleatorio $\vec{Z} = (Z_1,\, \ldots,\, Z_{n})^{\text{T}}$ si dice \emph{gaussiano standard} $n$\nbdash dimensionale se le variabili aleatorie $Z_1,\, \ldots,\, Z_{n}$ sono gaussiane standard indipendenti.

            Un vettore aleatorio $\vec{Z}$ gaussiano appena definito è \underline{assolutamente continuo} e ha densità: \[
                f_{\vec{Z}}(z_1,\, \ldots,\, z_{n}) = \frac{1}{(2\pi)^{n /2}} \cdot e^{-\frac{1}{2}\sum_{k=1}^{n} z_k^2}
            ;\] questo risultato si ottiene dall'indipendenza delle componenti di $\vec{Z}$: \[
            f_{\vec{Z}}(z_1,\, \ldots,\, z_{n}) = f_{Z_1}(z_1) \cdot \ldots \cdot f_{Z_n}(z_n) =
            \frac{1}{\sqrt{2\pi}} \cdot e^{- z_1^2 /2} \cdot \ldots \cdot \frac{1}{\sqrt{2\pi}} \cdot e^{- z_n^2 /2} = \frac{1}{(2\pi)^{n /2}} \cdot e^{-\frac{1}{2}\sum_{k=1}^{n} z_k^2}
            .\] 
        \end{defn}
        \begin{obsv}
            Media e matrice di covarianza di un vettore gaussiano standard si ottengono come:
            \begin{gather*}
                \text{E}(\vec{Z}) = \text{\O} = (0,\,0,\,\ldots); \\
                \vec{C}_{\vec{Z}} = \vec{I} \in \mathbb{R}^n \times \mathbb{R}^n.
            \end{gather*}
        \end{obsv}
        \begin{defn}[Normale]\label{defn:Vettore_gaussiano}
            Un vettore aleatorio $n$\nbdash dimensionale $\vec{X}$ è detto \emph{gaussiano} o normale se esistono una matrice $\vec{A} \in \mathbb{R}^n \times \mathbb{R}^m$, un vettore $\vec{\mu} \in \mathbb{R}^n$ e un vettore gaussiano \underline{standard} $\vec{Z} \in \mathbb{R}^m$ tali che: \[
            \vec{X} = \vec{A} \cdot \vec{Z} + \vec{\mu}
            .\] 
        \end{defn}
        \begin{obsv}\label{obsv:Vettore_gaussiano}
            Media e matrice di covarianza di un vettore gaussiano si ottengono come:
            \begin{gather*}
                \text{E}(\vec{X}) = \vec{A} \cdot \text{E}(\vec{Z}) + \vec{\mu} = \vec{\mu}; \\
                \vec{C}_{\vec{X}} = \vec{A} \cdot \vec{I} \cdot \vec{A}^{\text{\,T}} = \vec{A} \cdot \vec{A}^{\text{\,T}}.
            \end{gather*}
        \end{obsv}
        \begin{prty}
            Un vettore gaussiano $\vec{X} = \vec{A} \cdot \vec{Z} + \vec{\mu}$ ha densità su $\mathbb{R}^n$ se e solo se la matrice di covarianza $\vec{C} = \vec{A} \cdot \vec{A}^{\text{\,T}}$ è non-singolare; in tal caso la densità vale:
            \begin{equation}\label{eq:Densità_vettore_gaussiano}
                f_{\vec{X}}(\vec{x}) = \frac{1}{\sqrt{(2\pi)^n \cdot \det[\vec{C}]}} \cdot e^{-\frac{1}{2}(\vec{x}-\vec{\mu})^{\text{T}} \cdot \vec{C}^{-1}\cdot (\vec{x}-\vec{\mu})}
            .\end{equation}
        \end{prty}
        \begin{obsv}
            Consideriamo un vettore aleatorio assolutamente continuo $\vec{X}$ con densità \eqref{eq:Densità_vettore_gaussiano}, dove $\vec{C}$ è una matrice simmetrica definita positiva; allora possiamo estrarre la radice di $\vec{C}$ come: \[
            \exists \vec{A} \,:\, \vec{C} = \vec{A} \cdot \vec{A}^{\text{\,T}}
        .\] Prendiamo ora un vettore $\vec{Z} = \vec{A}^{\,-1}\cdot(\vec{X} - \vec{\mu})$, e calcoliamo la sua densità usando \eqref{eq:Trasformazioni_vettori_continui}:
        \begin{align*}
            f_{\vec{Z}}(\vec{z}) = f_{\vec{X}}(\vec{A} \cdot \vec{z} + \vec{\mu}) \cdot |\det(\vec{A})| &=
            \frac{|\det(\vec{A})|}{\sqrt{(2\pi)^n \cdot \det(\vec{A} \cdot \vec{A}^{\text{\,T}})}} \cdot 
            e^{-\frac{1}{2}(\vec{A}\cdot \vec{z} + \vec{\mu}-\vec{\mu})^{\text{T}} \cdot (\vec{A} \cdot \vec{A}^{\text{\,T}})^{-1} \cdot (\vec{A}\cdot \vec{z} +\vec{\mu}-\vec{\mu})} \\
            &= \frac{1}{\sqrt{(2\pi)^n}} \cdot e^{-\frac{1}{2}\, \vec{z}^{\text{\,T}} \cdot \vec{z}}
            \implies \vec{Z} \sim \mathcal{N}(0,\,\vec{I})
        .\end{align*}
        Abbiamo mostrato che $\vec{X} = \vec{A} \cdot \vec{Z} + \vec{\mu}$ è un vettore gaussiano di media $\vec{\mu}$ e matrice di covarianza $\vec{C}$
        \end{obsv}
        \begin{prty}
            Sia $\vec{X} = \vec{A} \cdot \vec{Z} + \vec{\mu}$ un vettore gaussiano $n$\nbdash dimensionale, 
            sia $\vec{C} = \vec{A} \cdot \vec{A}^{\text{\,T}}$ la matrice di covarianza di $\vec{X}$; 
            allora possiamo affermare che:
            \begin{enumerate}
                \item $c_{i,\,i} \in \vec{C}_{\vec{X}} \,:\, \begin{cases}
                    X_i \sim \mathcal{N}(\mu_i,\,c_{i,\,i}) & \text{se $c_{i,\,i} > 0$;} \\
                    P(X_i = \mu_i) = 1 & \text{se $c_{i,\,i} = 0$;}
                    \end{cases}$
                \item $\vec{G} \in \mathbb{R}^k \times \mathbb{R}^n \land \vec{h} \in \mathbb{R}^k
                    \implies \vec{Y} \coloneqq \vec{G} \cdot \vec{X} + \vec{h} \sim 
                    \mathcal{N}\left(\vec{G}\, \vec{\mu} + \vec{h},\, \vec{G}\, \vec{C}\, \vec{G}^{\text{\,T}}\right);$
                \item se $X_1,\, \ldots,\, X_{n}$ non sono correlate allora sono indipendenti.
            \end{enumerate}
        \end{prty}
        \begin{proof}
            \hfill
            \begin{enumerate}
                \item Dato che ogni componente $X_i$ del vettore aleatorio $\vec{X}$ si può esprimere come 
                    combinazione lineare di variabili aleatorie gaussiane indipendenti più una 
                    costante (dalla Definizione~\ref{defn:Vettore_gaussiano}); 
                    dall'Osservazione~\ref{obsv:Somma_gaussiane_indipendenti} possiamo dedurre che anche $X_i$ 
                    definita in tal modo è ancore una variabile aleatoria gaussiana.
                \item Scriviamo $\vec{Y}$ nel modo seguente: \[
                        \vec{G} \cdot \vec{X} + \vec{h} = \vec{G}(\vec{A} \cdot \vec{Z} + \vec{\mu}) + \vec{h}
                        = (\vec{G} \cdot \vec{A})\cdot \vec{Z} + (\vec{G} \cdot \vec{\mu} + \vec{h})
                    .\] Usando la Definizione~\ref{defn:Vettore_gaussiano} e 
                    l'Osservazione~\ref{obsv:Vettore_gaussiano} possiamo affermare che:
                    \begin{align*}
                        \text{E}(\vec{X}) = \vec{G}\, \vec{\mu} + \vec{h}, & &
                        \vec{C}_{\vec{X}} = \vec{G}\, \vec{C}\, \vec{G}^{\text{\,T}}
                    .\end{align*}
                \item Consideriamo una matrice di covarianza invertibile ($\vec{X}$ ha densità su 
                    $\mathbb{R}^n$); se le componenti $X_1,\, \ldots,\, X_{n}$ non sono correlate, allora 
                    $\vec{C}_{\vec{X}}$ è una matrice diagonale costruita come segue: \[
                        \vec{C}_{\vec{X}} = \text{diag}\left(\sigma_1^2 = \text{Var}(X_1),\, 
                        \ldots,\, \sigma_n^2 = \text{Var}(X_n)\right)
                    ;\] allora la densità di $\vec{X}$ si ottiene come segue:
                    \begin{align*}
                        f_{\vec{X}}(\vec{x}) &= \frac{1}{\sqrt{(2\pi)^n \cdot \det(\vec{C})}} \cdot e^{-\frac{1}{2}(\vec{x}-\vec{\mu})^{\text{T}}cd \vec{C}^{\,-1}\cdot(\vec{x}-\vec{\mu})} \\
                        &= \frac{1}{\sqrt{(2\pi)\cdot [\sigma_1^2\cdot \ldots\cdot \sigma_n^2]}} \cdot e^{-\frac{1}{2} \sum_{i=1}^{n} \left[\frac{x_i-\mu_i}{\sigma_i}\right]^2} \\
                        &= \prod_{i=1}^{n} \frac{1}{\sqrt{2\pi \sigma_i^2}} \cdot e^{-\frac{1}{2}\left[\frac{x_i-\mu_i}{\sigma_i}\right]^2} \\
                        &= \prod_{i=1}^{n} f_{X_i}(x_i)
                    .\end{align*}
                    Segue che le $X_i$ sono indipendenti. \qedhere
            \end{enumerate}
        \end{proof}
    \section{Distribuzione Gamma}
        \begin{defn}
            Una variabile aleatoria continua si dice avere distribuzione \emph{gamma} se la sua densità si ottiene dalla Definizione~\ref{defn:Densità_Gamma}:
            \begin{equation}\label{eq:Densità_Gamma_Eulero}
                f_{X}(x) = \begin{cases}
                    \frac{\lambda^\alpha}{\Gamma(\alpha)}\cdot x^{\alpha-1}\cdot e^{-\lambda\, x} & \text{se $x > 0$;} \\
                    0 & \text{se $x \leq 0$.}
                \end{cases}
            \end{equation}
            Nella precedente siano definiti i parametri $\alpha > 0,\, \lambda > 0$ e la funzione $\Gamma(\alpha)$ si chiama \emph{Gamma di Eulero} ed è definita come:
            \begin{align}\label{eq:Gamma_di_Eulero}
                \Gamma(\alpha) &\coloneqq \int_{0}^{\infty} \lambda^\alpha \cdot x^{\alpha -1}\cdot e^{-\lambda\, x}\, dx \notag\\
                &= \int_{0}^{\infty} y^{\alpha -1}\cdot e^{-y}\, dy & & [y=\lambda\cdot x]
            .\end{align}
        \end{defn}
        \begin{obsv}[Gamma di Eulero]
            Applicando la formula dell'integrazione per parti alla \eqref{eq:Gamma_di_Eulero}, nel caso in cui $\alpha > 1$, possiamo riscriverla come:
            \begin{align*}
                \Gamma(\alpha) &= \int_{0}^{\infty} y^{\alpha-1}\cdot e^{-y}\, dy 
                \overset{\scriptscriptstyle \text{P.P.} \textstyle}{=}
                \left[-y^{\alpha-1} \cdot e^{-y}\right]_{y=0}^{y=+\infty} + 
                \int_{0}^{\infty} (\alpha-1) \cdot y^{\alpha-2} \cdot e^{-y}\, dy \\
                &= (\alpha-1) \cdot \int_{0}^{\infty} y^{\alpha-2} \cdot e^{-y}\, dy \\
                &= (\alpha-1)\cdot \Gamma(\alpha-1)
            .\end{align*}
            Possiamo usare questo risultato per calcolare il valore della funzione Gamma sugli interi:
            \begin{align*}
                \Gamma(n) &= (n-1)\cdot \Gamma(n-1) \\
                &= (n-1)\cdot (n-2)\cdot \Gamma(n-2) \\
                &\;\; \vdots \\
                &= (n-1)! \cdot \Gamma(1)
            .\end{align*}
            Sapendo che per $n=1$ otteniamo: \[
                \Gamma(1) = \int_{0}^{\infty} e^{-y}\, dy = 1
            ,\] possiamo ottenere, per $n > 1,\, n \in \mathbb{N}$:
            \begin{equation}\label{eq:Gamma_di_Eulero_interi}
                \Gamma(n) = (n-1)!
            .\end{equation}
        \end{obsv}
        \begin{prty}[Momenti gamma]
            La generatrice dei momenti di una variabile aleatoria di tipo Gamma con parametri $\alpha,\, \lambda$ si ottiene come segue:
            \begin{align}\label{eq:Generatrice_momenti_Gamma}
                m_X(t) &\coloneqq \text{E}(e^{t\, X}) =
                \int_{0}^{\infty} e^{t\, X}\cdot \frac{\lambda^{\alpha}}{\Gamma(\alpha)}\cdot x^{\alpha-1}\cdot e^{-\lambda x}\, dx \notag\\
                &= \frac{\lambda^\alpha}{\Gamma(\alpha)}\cdot \int_{0}^{\infty} x^{\alpha-1} \cdot e^{-(\lambda-t)x}\, dx \notag\\
                &= \left(\frac{\lambda}{\lambda-t}\right)^\alpha \cdot 
                \frac{1}{\Gamma(\alpha)}\cdot 
                \underset{\Gamma(\alpha)}{\underbrace{\int_{0}^{\infty} y^{\alpha-1}\cdot e^{-y}\, dy}} 
                & & \left[y = (\lambda-t)x\right] \notag\\
                &= \left(\frac{\lambda}{\lambda-t}\right)^\alpha
            .\end{align}
        \end{prty}
        \begin{defn}
            Usando la funzione \eqref{eq:Generatrice_momenti_Gamma} possiamo ottenere valore atteso e varianza per una variabile aleatoria con distribuzione Gamma:
            \begin{itemize}
                \item $\text{E}(X) = m_X(0)^{\prime} = 
                    \frac{\alpha\cdot \lambda^{\alpha}}{(\lambda-0)^{\alpha+1}} = 
                    \frac{\alpha}{\lambda}$;
                \item $\text{E}(X^2) = m_X(0)^{\prime\prime} = 
                    \frac{\alpha(\alpha+1)\cdot \lambda^{\alpha}}{(\lambda-0)^{\alpha+2}} = 
                    \frac{\alpha(\alpha+1)}{\lambda^2}$;
                \item $\text{Var}(X) = \text{E}(X^2) - \text{E}(X)^2 = \frac{\alpha}{\lambda^2}$.
            \end{itemize}
        \end{defn}
        \begin{defn}[Somma di Gamma indipendenti]
            Consideriamo una successioni di variabili aleatorie indipendenti identicamente distribuite $X_1,\, \ldots,\, X_{n}$ tali che ciascuna di esse abbia densità Gamma  $X_i \sim \Gamma(\alpha_i,\, \lambda)$; allora la loro somma ha densità: \[
                X_1 + \ldots + X_{n} = \sum_{i=1}^{n} \left[X_i \sim \Gamma(\alpha_i,\, \lambda)\right] \sim \Gamma\left(\sum_{i=1}^n \alpha_i,\, \lambda\right)
            .\] Osservando che: \[
            \Gamma(1,\, \lambda) \sim \mathcal{E}(\lambda)
        ,\] possiamo affermare che, presa la successione $Y_1,\, \ldots,\, Y_{n}$ dove $Y_i \sim \mathcal{E}(\lambda)$, la distribuzione della somma delle $Y_i$ si ottiene come: \[
        Y_1 + \ldots + Y_{n} = \sum_{i=1}^n \left[Y_i \sim \mathcal{E}(1,\, \lambda)\right] \sim \Gamma(n,\, \lambda)
        .\] 
        \end{defn}
    \section{Distribuzioni derivate dalla Normale}
        \subsection{Distribuzione Chi-Quadro}
            \begin{defn}\label{defn:Distribuzione_chi_quadro}
                Prendiamo una successione $Z_1,\, \ldots,\, Z_{n}$ di variabili aleatorie gaussiane standard indipendenti; allora la somma dei loro quadrati $X \coloneqq Z^2_1 + \ldots + Z^2_{n}$ è una variabile aleatoria chiamata \emph{chi-quadro} con $n$ gradi di libertà, e si indica come: \[
                    X \sim \chi^2(n)
                .\]
            \end{defn}
            \begin{obsv}[Riproducibilità]
                Se $X_1,\,X_2$ sono due variabili aleatorie con densità chi-quadro \underline{indipendenti}, rispettivamente a $n_1,\,n_2$ gradi di libertà, la loro somma ha la seguente densità: \[
                    W = X_1 + X_2 \sim \chi^2_{n_1} + \chi^2_{n_2} = \chi^2_{n_1+n_2}
                .\] Possiamo provare questo risultato usando direttamente la Definizione~\ref{defn:Distribuzione_chi_quadro}: sapendo che la somma di $X_1$ e $X_2$ è la somma di $n_1$ con $n_2$ normali standard indipendenti, è evidente che $W$ corrisponde a una chi-quadro con la somma dei gradi di libertà $n_1+n_2$.
            \end{obsv}
            \begin{prty}[Momenti chi-quadro]\label{prty:Generatrice_momenti_chi_quadro}
                Consideriamo il caso con $n=1$ gradi di libertà, e chiediamoci quanto vale la generatrice dei momenti di $X \sim \chi^2_1$, assumendo: \[
                    X = Z^2 \land Z \sim \mathcal{N}(0,\,1)
                ;\] allora otteniamo la generatrice dei momenti come:
                \begin{align*}
                    m_{X}(t) &= m_{Z^2}(t) =
                    \int_{-\infty}^{\infty} e^{t\, x^2}\cdot f_{Z}(x)\, dx = 
                    \int_{-\infty}^{\infty} e^{t\, x^2}\cdot \frac{1}{\sqrt{2\pi}}\, e^{-x^2 /2}\, dx \\
                    &= \int_{-\infty}^{\infty} \frac{1}{\sqrt{2\pi}}\cdot e^{-(1-2t)\frac{x^2}{2}}\, dx \\
                    &= (1-2t)^{-1 /2} \cdot \int_{-\infty}^{\infty} \underset{\sim\mathcal{N}(0,\,\bar{\sigma}^2)}{\underbrace{\frac{1}{\bar{\sigma}\, \sqrt{2\pi}} \cdot e^{-\frac{x^2}{2\bar{\sigma}^2}}}}\, dx & & \text{per } \bar{\sigma} = (1-2t)^{-1 /2} \\
                    &= (1-2t)^{-1 /2}
                .\end{align*}
                Nel caso più generale con $n$ gradi di libertà ($X \sim \chi^2(n)$), si ha:
                \begin{align}\label{eq:Generatrice_momenti_chi_quadro}
                    m_{X}(t) &= \text{E}(e^{t\,X}) = \text{E}(e^{t\,Z_1^2 + \ldots + t\,Z_n^2}) \notag\\
                             &= \text{E}(e^{t\,Z_1^2}\cdot \ldots \cdot e^{t\,Z_n^2}) \notag\\
                             &=\prod_{i=1}^{n} \text{E}(e^{t\,Z_i^2}) = (1-2t)^{-n /2} & & \text{per $Z_i$ indipendenti\notag}\\
                             &= \left(\frac{1}{1-2t}\right)^{n /2} = \left(\frac{1 /2}{1 /2 -t}\right)^{n /2} & & \text{dal caso $n=1$}
                .\end{align}
            \end{prty}
            \begin{defn}[Definizione alternativa]
                La densità Gamma di parametri $\alpha = n /2$ e $\lambda = 1 /2$  con $n \geq 1$ intero, è anche indicata come densità \emph{chi-quadro} con $n$ gradi di libertà.

                Possiamo provarlo direttamente dal risultato della Proprietà~\ref{prty:Generatrice_momenti_chi_quadro}: infatti la generatrice dei momenti della distribuzione chi-quadro ad $n$ gradi di libertà \eqref{eq:Generatrice_momenti_chi_quadro} corrisponde alla generatrice della distribuzione Gamma nel caso indicato sopra \eqref{eq:Generatrice_momenti_Gamma}.
            \end{defn}
        \subsection{Distribuzione \emph{t}}
            \begin{defn}
                Consideriamo le variabili aleatorie indipendenti $Z \sim \mathcal{N}(0,\,1)$ e $C_n \sim \chi^2(n)$; allora la variabile aleatoria definita come segue:
                \begin{equation}\label{eq:Distribuzione_t}
                    T_n \coloneqq \frac{Z}{\sqrt{C_n /n}}
                .\end{equation}
                ha distribuzione chiamata \emph{t di Student} con $n$ gradi di libertà, e si indica con: \[
                    T_n \sim t(n)
            .\]
            \end{defn}
            \begin{prty}\label{prty:Distribuzione_t}
                Sia data una variabile aleatoria $T_n \sim t(n)$, allora possiamo affermare che:
                \begin{enumerate}
                    \item la distribuzione \emph{t} è simmetrica rispetto all'ascissa ($x=0$), ovvero $T_n \sim -T_n$;
                    \item per $n$ adeguatamente grande ($n \geq 100$) possiamo approssimare $T_n \sim \mathcal{N}(0,\,1)$;
                    \item per $n>2$ otteniamo $\text{E}(T_n)=0,\, \text{Var}(T_n)=\frac{n}{n-2}$.
                \end{enumerate}
            \end{prty}
            \begin{proof}
                \hfill
                \begin{enumerate}
                    \item Usando la definizione della distribuzione \emph{t} sappiamo che: \[
                        Z \sim \mathcal{N}(0,\,1) \sim -Z \implies
                        -T_n = \frac{-Z}{\sqrt{C_n /n}} \sim \frac{Z}{\sqrt{C_n /n}} = T_n
                    .\] 
                    \item Dalla funzione di ripartizione di $T_n$ otteniamo: \[
                            F_{T_n}(x) \coloneqq P(T_n \leq x) = P\left(\frac{Z}{\sqrt{C_n /n}} \leq x\right)
                        ;\] applicando il Teorema~\ref{thm:Legge_grandi_numeri_forte} possiamo affermare che, per $n$ grande, $X /n \rightarrow \text{E}(Z_i^2)=1$; quindi: \[
                        \lim_{n \to \infty} F_{T_n}(x) = 
                        \lim_{n \to \infty} P\left(\frac{Z}{\sqrt{C_n /n}} \leq x\right) \rightarrow
                        P\left(\frac{Z}{1} \leq x\right) = \Phi(x) = \frac{1}{\sqrt{2\pi}} \cdot \int_{-\infty}^{x} e^{-u^2 /2}\, du
                        .\qedhere\]
                \end{enumerate}
            \end{proof}
        \subsection{Distribuzione \emph{F}}
        \begin{defn}
            Consideriamo due variabili aleatorie indipendenti $C_n \sim \chi^2(n)$ e $C_m \sim \chi^2(m)$; allora la variabile aleatoria definita in modo seguente:
            \begin{equation}\label{eq:Distribuzione_F}
                F_{n,\,m} \coloneqq \frac{C_n /n}{C_m /m}
            ,\end{equation}
            ha una distribuzione chiamata \emph{F di Fisher} con $n$ e $m$ gradi di libertà, e la indichiamo con: \[
                F_{n,\,m} = \mathcal{F}(n,\,m)
            .\]
        \end{defn}
        \begin{prty}
            Sia data una variabile aleatoria con densità di Fisher $F \sim \mathcal{F}(n,\,m)$; allora possiamo affermare che $F \geq 0$, essendo quoziente di due variabili aleatorie con distribuzione chi-quadro (a loro volta somma di quadrati, quindi positive per definizione);
        \end{prty}
    \section{Quantili}
        \begin{defn}
            Chiamiamo \emph{quantile} di ordine $\alpha$, e lo indichiamo con $q_\alpha$ per 
            $\alpha \in [0,\,1]$, il valore ottenuto dalla funzione di ripartizione $F_W(w)$ di una 
            variabile aleatoria continua $W$ come: \[
                F_W(q_\alpha) = P(W \geq q_\alpha) = \alpha
            .\] La distribuzione viene suddivisa da $q_\alpha$ in due intervalli proporzionali ad $\alpha$ e a $(1-\alpha)$.
        \end{defn}
        \subsection{Quantili della normale standard}
            \begin{defn}
                Data la variabile aleatoria $Z \sim \mathcal{N}(0,\,1)$ chiamiamo quantile di coda sinistra di ordine $\alpha$ il valore definito come: \[
                    P(Z > z_\alpha) = 1 - \Phi(z_\alpha) = \alpha
                .\]
            \end{defn}
            \begin{obsv}
                Una variabile aleatoria con distribuzione normale standard avrà probabilità inferiore o uguale a $z_\alpha$ nel $[100\cdot (1-\alpha)]\%$ dei casi. Inoltre usando la definizione appena enunciata possiamo scrivere: \[
                    z_\alpha \coloneqq \Phi^{-1}(1-\alpha)
                .\]
            \end{obsv}
        \subsection{Quantili della chi-quadro}
            \begin{defn}
                Data la variabile aleatoria $C \sim \chi^2(n)$ chiamiamo quantile di coda destra 
                di ordine $\alpha$ il valore definito come: \[
                    P(C > \chi^2_{\alpha,\,n}) = \alpha
                .\]
            \end{defn}
        \subsection{Quantili della \emph{t}}
            \begin{defn}
                Data la variabile aleatoria $T \sim t(n)$ chiamiamo quantile di coda destra 
                di ordine $\alpha$ il valore definito come: \[
                    P(T > t_{\alpha,\,n}) = \alpha
                .\]
            \end{defn}
            \begin{obsv}
                Data la simmetria della densità \emph{t} rispetto all'asse $x=0$, possiamo scrivere:
                \begin{align*}
                    \alpha &= P(-T \geq t_{\alpha,\,n})
                    = P(T \leq -t_{\alpha,\,n}) \\
                    &= 1 - P(T > -t_{\alpha,\,n})
                .\end{align*}
                Segue che:
                \begin{equation}\label{eq:Simmatria_quantile_distribuzione_t}
                    P(T \geq -t_{\alpha,\,n}) = 1 - \alpha \implies -t_{\alpha,\,n} = t_{1-\alpha,\,n}
                .\end{equation}
            \end{obsv}
        \subsection{Quantili della \emph{F}}
            \begin{defn}
                Data la variabile aleatoria $F \sim \mathcal{F}(n,\,m)$ chiamiamo quantile di coda destra 
                di ordine $\alpha$ il valore definito come: \[
                    P(F > f_{\alpha,\,n,\,m}) = \alpha
                .\]
            \end{defn}
            \begin{obsv}
                Analizziamo il caso di un quantile di ordine simmetrico ($1-\alpha$):
                \begin{align*}
                    \alpha &= P\left(\frac{C_n /n}{C_m /m} > f_{\alpha,\,n,\,m}\right) =
                    P\left(\frac{C_m /m}{C_n /n} < \frac{1}{f_{\alpha,\,n,\,m}}\right) \\
                           &= 1 - P\left(\frac{C_m /m}{C_n /n} \geq \frac{1}{f_{\alpha,\,n,\,m}}\right)
                .\end{align*}
                Segue che:
                \begin{equation}\label{eq:Simmetria_quantile_distribuzione_F}
                    P\left(\frac{C_m /m}{C_n /n} \geq f_{1-\alpha,\,n,\,m}\right) = 1-\alpha \implies
                    \frac{1}{f_{\alpha,\,n,\,m}} = f_{1-\alpha,\,n,\,m}
                .\end{equation}
            \end{obsv}
