%! TEX root = main.tex
% Stile del documento
\documentclass[12pt,openany]{amsbook}
\usepackage[T1]{fontenc}
\usepackage[utf8]{inputenc}
\usepackage[a4paper]{geometry}
\geometry{verbose,tmargin=1.8cm,bmargin=2cm,lmargin=2cm,rmargin=2cm,headheight=1.8cm,headsep=1cm,footskip=1cm}
% Localizzazione (traduce nomi automatici nella lingua selezionata)
\usepackage[italian]{babel}
% Altri pacchetti per la formattazione
\usepackage{lmodern}
\usepackage{hyphenat}
\usepackage{amsmath}
\usepackage{amsthm}
\usepackage{amssymb}
\usepackage{mathrsfs}
\usepackage{mathtools}
\usepackage{cancel}
\usepackage{enumerate}
\usepackage{graphicx}
 % Impostazioni dei metadati del PDF
\usepackage[unicode=true, bookmarks=true, bookmarksnumbered=true, bookmarksopen=true, bookmarksopenlevel=1, breaklinks=false, pdfborder={0 0 0}, pdfborderstyle={}, backref=page, colorlinks=false]
{hyperref}
\hypersetup{
    pdftitle={Appunti di Probbilità e Statistica},
    pdfauthor={Lorenzo Prosseda},
    pdfsubject={Corso di Probabilità e Statistica del prof. Barchielli, Politecnico di Milano 2018\---2019},
    pdfkeywords={probabilità, statistica}
}
% Percorso della cartella contenente le immagini
\graphicspath{ {../img/} }
% Impostazioni di stile per gli ambienti teorema, definizione, esempio, ecc
\theoremstyle{plain}
\newtheorem{thm}{Teorema}[section] % resetta numerazione ad ogni capitolo
\renewcommand{\thethm}{\arabic{chapter}.\arabic{section}.\arabic{thm}} % numerazione ambienti teorema
\numberwithin{equation}{section} % numerazione equazioni
% Definizione ambienti personalizzati
\theoremstyle{definition}
\newtheorem{defn}[thm]{Definizione}
\newtheorem{prty}[thm]{Proprietà}
\newtheorem{exmp}[thm]{Esempio}
\newtheorem{obsv}[thm]{Osservazione}
\theoremstyle{remark}
\newtheorem{note}[thm]{Nota}
% Numerazione personalizzata sezioni e sottosezioni
\renewcommand\thesection{\thechapter.\arabic{section}}
\renewcommand\thesubsection{\thesection.\arabic{subsection}}
% Comando trattino "non spezzante" evita sillabazione sul trattino
\newcommand\nbdash{\nobreakdash-\hspace{0pt}}
% Non spezzare e seguenti parole:
\hyphenation{Proprietà}
\hyphenation{Definizione}
\hyphenation{Osservazione}
\hyphenation{Equazione}
\hyphenation{Teorema}
% Solleva posizione lettera greca chi minuscola
\usepackage{letltxmacro}
\LetLtxMacro{\lowchi}{\chi}
\renewcommand\chi{\text{\raisebox{2pt}{$\lowchi$}}}

% Includi solo il capitolo in corso
\includeonly{chap6}

% In questo ambiente risiede il contenuto del documento
\begin{document}
    % Pagina del titolo
    %! TEX root = main.tex

\begin{titlepage}
    \centering % centra tutto in questo ambiente
    \vspace*{\stretch{0.8}}
    \Huge\textbf{Appunti di Probabilità e Statistica}\\
    \vspace*{\stretch{0.6}}
    \Large\emph{Lorenzo Prosseda}\\
    \bigskip
    \large a.a. 2018\---2019 \\
    \vspace*{\stretch{5.0}}
    \begin{figure}[h] % inserisci logo ed estratto della licenza GFDL
        \includegraphics[width=2cm]{gfdl-logo.pdf}
        \bigskip \\
        \small\emph{Copyright \textcopyright\,2019 Lorenzo Prosseda. Permission is granted to copy, distribute and/or modify this document under the terms of the GNU Free Documentation License, Version 1.3 or any later version published by the Free Software Foundation; with no Invariant Sections, no Front-Cover Texts, and no Back-Cover Texts. A copy of the license is included in the file called ``LICENSE''.}
    \end{figure}
    \vspace*{\stretch{1.0}}
\end{titlepage}


    % Genera indice di capitoli e sezioni
    \tableofcontents

    % Includi file dei capitoli
    \part{Probabilità}
    %! TEX root = main.tex
% Capitolo 1

\chapter{Teoria della Probabilità}
    \section{Spazio dei campioni}
        \begin{defn}
            Sia dato un esperimento \textit{aleatorio} (impossibile prevederne risultato), i cui risultati siano rappresentati da $\omega \in \Omega$; chiamiamo l'insieme $\Omega$:
            \begin{itemize}
                \item spazio campionario;
                \item spazio dei campioni;
                \item spazio degli eventi \underline{elementari};
                \item spazio degli esiti;
            \end{itemize}
            diciamo inoltre che $\Omega$ è relativo all'esperimento effettuato, e chiamiamo i suoi elementi $\omega$ \textit{eventi \underline{elementari}}.
        \end{defn}
        \section{Eventi}\label{sec:Eventi}
        \begin{defn}
            I sotto-insiemi dello spazio campionario e le loro combinazioni (in termini di eventi elementari) tramite operatori logici di unione ($\cup$), intersezione ($\cap$) e negazione ($^{\text{C}}$) sono chiamati \textit{eventi} (non sono più elementari).
        \end{defn}
        \begin{obsv}
            Gli eventi \textit{elementari} sono rappresentati da sottoinsiemi di $\Omega$ di cardinalità 1: un evento elementare $E$ è definito come $\{\omega_E\} \in \Omega \land |\{\omega_E\}| = 1$.
        \end{obsv}
        \begin{defn}
            Gli eventi $E_i$ possono essere rappresentati con sottoinsiemi dello spazio campionario $\Omega$, dunque essi formano una \textit{famiglia} o \textit{collezione} di sottoinsiemi di $\Omega$, che indichiamo con $\mathscr{F}$.

            Questo implica che un evento che faccia parte di questa famiglia di sottoinsiemi contenga lo spazio campionario ($E \in \mathscr{F} \implies E \subset \Omega$).

            Diremo che si è \textit{verificato} un evento $E$ nel contesto di un esperimento aleatorio, se almeno uno degli esiti è contenuto in esso ($\omega \in E$).
        \end{defn}
        \begin{defn}
            Dati due eventi $E$ ed $F$, sono definite le operazioni di unione $E \cup F$ (gli eventi elementari in $E$ o in $F$) e intersezione $E \cap F$ (gli eventi elementari presenti sia in $E$ che in $F$).
            \begin{itemize}
                \item Gli eventi che non contengono alcun evento elementare sono chiamati \textit{eventi vuoti} e si indicano con \O. Se due eventi non hanno eventi elementari in comune ($E \cap F = \text{\O}$), essi si dicono \textit{disgiunti}.
                \item Per ogni evento $E \in \mathscr{F}$ è definito il complementare $E^{\text{C}}$ come l'insieme degli eventi elementari di $\mathscr{F}$ che non stanno in $E$.
                \item Se tutti gli eventi elementari di un evento $E$ sono anche in un evento  $F$, diremo che $E$ \textit{è contenuto} in  $F$ ($E \subset F$); se vale anche l'inverso $E \supset F$ allora diremo che i due eventi sono \textit{uguali} ($E \equiv F$).
            \end{itemize}
            Indichiamo l'unione di $n$ eventi con  $\bigcup_{k=1}^{n} E_k$ e l'intersezione di  $n$ eventi con $\bigcap_{k=1}^{n} E_k$.
        \end{defn}
        \begin{defn}
            Sia $\Omega$ uno spazio campionario e $\mathscr{F}$ una famiglia di sottoinsiemi di $\Omega$; diremo che $\mathscr{F}$ è un'\textit{algebra di sottoinsiemi} di $\Omega$ se soddisfa le seguenti proprietà:
            \begin{enumerate}
                \item $\Omega \in \mathscr{F}$;
                \item $E \in \mathscr{F} \implies E^{\text{C}} \coloneqq \Omega \backslash E \in \mathscr{F}$;
                \item $E,\,F \in \mathscr{F} \implies \{E \cup F\} \in \mathscr{F}$.
            \end{enumerate}
        \end{defn}
        \begin{defn}
            Sia $\Omega$ uno spazio campionario e $\mathscr{F}$ una famiglia di sottoinsiemi di $\Omega$; diremo che $\mathscr{F}$ è una \textit{$\sigma$\nbdash algebra di sottoinsiemi} di $\Omega$ se soddisfa le seguenti proprietà:
            \begin{enumerate}
                \item $\Omega \in \mathscr{F}$;
                \item $E \in \mathscr{F} \implies E^{\text{C}} \coloneqq \Omega \backslash E \in \mathscr{F}$;
                \item $E_1,\,E_2,\,\ldots \in \mathscr{F} \implies \bigcup_{k=1}^{+\infty}E_k \in \mathscr{F}$.
            \end{enumerate}
        \end{defn}
        \begin{defn}
            La coppia di spazio campionario e famiglia di suoi sottoinsiemi ($\sigma$\nbdash algebra) $(\Omega,\,\mathscr{F})$ è chiamata \textit{spazio probabilizzabile}.
        \end{defn}
    \section{Spazio di probabilità}
        \begin{defn}\label{defn:Probabilità_impostazione_assiomatica}
            Sia dato uno spazio probabilizabile $(\Omega,\,\mathscr{F})$; chiamiamo \textit{probabilità} su $(\Omega,\,\mathscr{F})$ una funzione $P$ su $\mathscr{F}$ tale che:
            \begin{enumerate}
                \item $\forall E \in \mathscr{F}\,:\,P(E) \geq 0$;
                \item $P(\Omega) = 1$;
                \item $\forall h,k\,:\,h \neq k \implies \forall E \in \mathscr{F}\,:\,E_h \cap E_k = 0$ ($\sigma$\nbdash additività o \textit{additività completa}).
            \end{enumerate}
            La terna $(\Omega,\,\mathscr{F},\,P)$ si chiama \textit{spazio di probabilità} (impostazione assiomatica).
        \end{defn}
        \begin{prty}\label{prty:Spazio_di_probabilità}
            Sia $(\Omega,\,\mathscr{F},\,P)$ uno spazio di probabilità; allora vale:
            \begin{enumerate}
                \item $P(\text{\O}) = 0$ (\textit{evento impossibile});
                \item $\forall h,k\,:\,h \neq k \implies \forall E \in \mathscr{F}\,:\,E_h \cap E_k = \text{\O} \implies P(\bigcup_{k=1}^{n} E_k) = \sum_{k=1}^{n} P(E_k)$ (\textit{additività finita}).
            \end{enumerate}
        \end{prty}
        \begin{proof}
            \hfill
            \begin{enumerate}
                \item $\forall k \in [1,\,n]\,:\,E_k \coloneqq \text{\O} \implies \bigcup_{k=1}^{+\infty} E_k = \text{\O}$, inoltre $E_1,\,\ldots,\,E_n$ è una successione di eventi disgiunti a coppie; per l'assioma $(3)$ della Definizione~\ref{defn:Probabilità_impostazione_assiomatica} vale inoltre: \[
                        P\left(\text{\O}\right) = P\left(\bigcup_{k=1}^{+\infty} E_k\right) = \sum_{k=1}^{+\infty} P\left(\text{\O}\right)
                .\] 
            \item Se $\forall k \in [n+1,\,\infty)\,:\,E_k = \text{\O}$, allora $E_1,\,E_2,\,\ldots$ è una successione di eventi disgiunti a coppie, e vale $\bigcup_{k=1}^{+\infty} E_k = \bigcup_{k=1}^{n} E_k$; per l'assioma $(3)$ della Definizione~\ref{defn:Probabilità_impostazione_assiomatica} vale inoltre: \[
                    P(\bigcup_{k=1}^{n} E_k) = (\bigcup_{k=1}^{+\infty} E_k) = \sum_{k=1}^{n} E_k + \underset{E_k = \text{\O}}{\underbrace{\cancel{\sum_{k=n+1}^{+\infty} E_k}}} = \sum_{k=1}^{n} E_k
            . \qedhere\]
            \end{enumerate}
        \end{proof}
    \section{Definizione assiomatica di Probabilità}
        \begin{defn}
            Se assegnamo a ogni evento in uno spazio di probabilità $E_k \in (\Omega,\,\mathscr{F},\,P)$ un valore $x = P(E_k)$, chiamiamo probabilità dell'evento $E$; esso deve rispettare i seguenti assiomi, dedotti dalla Definizione~\ref{defn:Probabilità_impostazione_assiomatica}:
            \begin{enumerate}[I)]
                \item $0 \leq P(E_k) \leq 1$;
                \item $P(\Omega) = 1$;
                \item  se gli eventi dello spazio di probabilità sono disgiunti a coppie otteniamo: \[
                \forall n \in [1,\,\infty)\,:\,P(\bigcup_{k=1}^{n} E_k) = \sum_{k=1}^{n} P(E_k)
                .\] 
            \end{enumerate}
        \end{defn}
    \section{Proprietà della funzione Probabilità}
    \begin{prty}\label{prty:Proprietà_funzione_probabilità}
            Sia $(\Omega,\,\mathscr{F},\,P)$ uno spazio di probabilità; su di esso possiamo osservare le seguenti proprietà:
            \begin{enumerate}
                \item $E \in \mathscr{F} \implies P(E^{\text{C}}) = 1 - P(E)$ (\textit{probabilità del complementare});
                \item $E \in \mathscr{F} \implies P(E) \leq 1$;
                \item $E,F \in \mathscr{F} \land F \subset E \implies P(E \backslash F) = P(E) - P(F)$;
                \item $E,F \in \mathscr{F} \land F \subset E \implies P(F) \leq P(E)$ (\textit{monotonia});
                \item $E,F \in \mathscr{F} \implies P(E \cup F) = P(E) + P(F) - P(E \cap F)$ (\textit{probabilità dell'unione}).
            \end{enumerate}
        \end{prty}
        \begin{proof}
            \hfill
            \begin{enumerate}
                \item Osserviamo che, per l'assioma $(2)$ della Definizione~\ref{defn:Probabilità_impostazione_assiomatica}, vale $\Omega = E \cup E^{\text{C}}$ e sapendo che un evento e il suo complementare sono disgiunti: \[
                    1 = P(\Omega)= P(E) + P(E^{\text{C}}) \implies P(E^{\text{C}}) = 1 - P(E)
                .\] 
                \item Dalla precedente considerazione sappiamo che $P(E) = 1 - P(E^{\text{C}})$; aggiungendo l'assioma $(1)$ della Definizione~\ref{defn:Probabilità_impostazione_assiomatica} ($P(E^{\text{C}}) \geq 0$) segue che deve valere $P(E) \leq 1$.
                \item Vale $F \subset E \implies E = (E \backslash F) \cup F$ da cui deduciamo, per l'assioma $(2)$ della Proprietà~\ref{prty:Spazio_di_probabilità}, la seguente scrittura: \[
                    P(E) = P(E \backslash F) + P(F) \implies P(E \backslash F) = P(E) - P(F)
                .\] 
                \item Dalla precedente ricaviamo immediatamente $P(E) = P(E \backslash F) + P(F)$ e  $P(F) = P(E) - P(E \backslash F)$; sapendo dall'assioma $(1)$ della Definizione~\ref{defn:Probabilità_impostazione_assiomatica} che tutti e tre i termini dell'uguaglianza sono positivi o al più nulli, abbiamo necessariamente $P(F) \leq P(E)$.
                \item Scriviamo l'unione disgiunta dei due eventi $E$ ed  $F$ nel modo seguente:  \[
                    E \cup F = (E \cap F^{\text{C}}) \cup (E \cap F) \cup (E^{\text{C}} \cap F)
                .\] Per il punto $(2)$ della Proprietà~\ref{prty:Spazio_di_probabilità} la probabilità dell'unione vale: \[
                    P(E \cup F) = P(E \cap F^{\text{C}}) + P(E \cap F) + P(E^{\text{C}} \cap F)
                .\] Riscriviamo la precedente come:
                \begin{align*}
                    P(E \cup F) + P(E \cap F) &= \overset{P(E)}{\overbrace{P(E \cap F^{\text{C}}) + P(E \cup F)}} + \overset{P(F)}{\overbrace{P(E \cup F) + P(E^{\text{C}} \cap F)}} \\ {}&= P(E) + P(F)
                .
                \end{align*}
                Per ottenere le semplificazioni evidenziate con le graffe abbiamo usato le proprietà delle operazioni tra eventi (§~\ref{sec:Eventi}). \qedhere
            \end{enumerate}
        \end{proof}
    \section{Spazi finiti e numerabili}
    \begin{prty}\label{prty:Spazi_numerabili}
        Sia $\Omega$ uno spazio campionario \underline{numerabile} e  $\{\omega_1,\,\omega_2,\,\ldots\}$ una numera\-zione dei suoi punti (eventi elementari); scegliamo come $\sigma$\nbdash algebra  $\mathscr{F}$ l'insieme di tutti i sottoinsiemi (\textit{insieme delle parti}) dello spazio campionario  $\mathscr{P}(\Omega)$; si ha che:
            \begin{enumerate}
                \item Ogni probabilità $P({\omega_k})$ di un evento $E_k = \{\omega_k\} \in \mathscr{F}$ su uno spazio probabiliz\-zabile $(\Omega,\,\mathscr{F})$ individua una successione di numeri reali $p_1,\,p_2,\,\ldots$ che soddisfano la Proprietà~\ref{prty:Spazi_numerabili} se scriviamo che $\forall k \in [1,\,\infty)\,\:\,P({\omega_k}) = p_k$;
                \item Data una successione $p_1,\,p_2,\,\ldots$ che soddisfa la Proprietà~\ref{prty:Spazi_numerabili} esiste un'unica probabilità su $(\Omega,\,\mathscr{F})$ tale che  $P({\omega_k}) = p_k$ per ogni  $k$; essa è data da: \[
                        \forall E \subset \Omega\,:\,P(E) = \sum_{k\,:\,\omega_k \in E} p_k
                .\] 
            \end{enumerate}
        \end{prty}
        \begin{obsv}
            Nel caso di $\Omega$ spazio campionario \underline{finito} e numerabile, la Proprietà~\ref{prty:Spazi_numerabili} continua a valere, rispetto alla cardinalità di $\Omega$.
        \end{obsv}
    \section{Probabilità condizionata}
    \begin{defn}\label{defn:Probabilità_condizionata}
            Sia dato uno spazio di probabilità $(\Omega,\,\mathscr{F},\,P)$ e un evento $F \in \mathscr{F}$ tale che $P(F) > 0$; preso un qualsiasi altro evento $E \in \mathscr{F}$, si chiama \textit{probabilità condizionata} dell'evento $E$ dato il verificarsi di $F$:
            \begin{equation}\label{eq:Formula_probabilità_condizionata}
                P(E|F) = \frac{P(E \cap F)}{P(F)}
            .
            \end{equation}
        \end{defn}
    \section{Formula delle probabilità totali}
        \begin{defn}\label{defn:Probabilità_totali}
            Consideriamo uno spazio di probabilità $(\Omega,\,\mathscr{F},\,P)$ e una partizione finita $F_1,\,\ldots,\,F_n \in \mathscr{F}$ di $\Omega$; valga inoltre $\bigcup_{k=1}^{n} F_k = \Omega$ e infine $\forall h,k \in [1,\,n] \,:\, F_h \cap F_k = \text{\O} \land P(F_k) > 0$.
            Allora per qualunque evento $E \in \mathscr{F}$ la sua probabilità è definita come:
            \begin{equation}\label{eq:Formula_probabilità_totali}
                P(E) = \sum_{k=1}^{n} P(E|F_k) \cdot P(F_k)
            . 
            \end{equation}
        \end{defn}
        \begin{proof}
            Prendiamo l'evento $E \in \mathscr{F}$; dalle considerazioni fatte sopra abbiamo la seguente implicazione:  \[
                \Omega = \bigcup_{k=1}^{n} F_k \land E \subset \Omega \implies E = E \cap \Omega = \bigcup_{k=1}^{n} (E \cap F_k)
            .\] 
            Inoltre, poiché gli eventi $F_1,\,\ldots,\,F_n$ sono disgiunti a coppie, la precedente $\bigcup_{k=1}^{n} (E \cap F_k)$ è un'unione disgiunta e applicando il punto $(2)$ della Proprietà~\ref{prty:Spazio_di_probabilità} otteniamo: \[
                P(E) = \sum_{k=1}^{n} P(E \cap F_k) = \sum_{k=1}^{n} P(E|F_k) \cdot P(F_k)
            .\] 
            Abbiamo così ottenuto la formula \eqref{eq:Formula_probabilità_totali}.
        \end{proof}
        \begin{obsv}
            La formula delle probabilità totali è utile quando le condizioni di preparazione di un esperimento aleatorio sono anch'esse casuali, e determinano una partizione dello spazio di probabilità dell'esperimento.
        \end{obsv}
    \section{Formula di Bayes}
        \begin{defn}\label{defn:Formula_Bayes}
            Consideriamo uno spazio di probabilità $(\Omega,\,\mathscr{F},\,P)$ e una partizione finita $F_1,\,\ldots,\,F_n \in \mathscr{F}$ di $\Omega$ tale che $\forall k \in [1,\,n] \,:\, P(F_k) > 0$; se abbiamo un evento $E \in \mathscr{F}$ per il quale $P(E) > 0$ allora otteniamo:
            \begin{align}\label{eq:Formula_Bayes}
                P(F_h|E) &= \frac{P(E|F_h) \cdot P(F_h)}{\sum_{k=1}^{n} P(E|F_k) \cdot P(F_k)} & h &= 1,\,\ldots,\,n
            .
            \end{align}
        \end{defn}
        \begin{proof}
            Usando \eqref{eq:Formula_probabilità_condizionata} possiamo scrivere: \[
                P(F_h|E) = \frac{P(F_h \cap E)}{P(E)} = \frac{P(E|F_h) \cdot P(F_h)}{P(E)}
            .\] 
            Applicando al denominatore \eqref{eq:Formula_probabilità_totali} otteniamo proprio la scrittura \eqref{eq:Formula_Bayes}.
        \end{proof}
        \begin{obsv}
            La formula di Bayes è utile quando possediamo informazioni a posteriori su un esperimento aleatorio e vogliamo determinare le condizioni entro le quali si sia verificato un certo evento.
        \end{obsv}
    \section{Formula di moltiplicazione}
        \begin{defn}\label{defn:Formula_moltiplicazione}
            Consideriamo uno spazio di probabilità $(\Omega,\,\mathscr{F},\,P)$ e una successione di eventi al suo interno $E_1,\,\ldots,\,E_n \in \mathscr{F}$, per i quali valga $P(\bigcap_{k=1}^{n-1} E_k) > 0$; allora possiamo scrivere:
            \begin{equation}\label{eq:Formula_moltiplicazione}
                P(E_1 \cap \dotsm \cap E_n) = P(E_1) \cdot P(E_2|E_1) \cdot P(E_3|E_2 \cap E_1) \cdot \ldots \cdot P(E_n|E_1 \cap \dotsm \cap E_{n-1})
            .
            \end{equation}
        \end{defn}
        \begin{proof}
            Dato che $(E_1 \cap \dotsm \cap E_{n-1}) \subset (E_1 \cap \dotsm \cap E_{n-2}) \subset \dotsm \subset E_1$ usando il punto $(4)$ della Proprietà~\ref{prty:Proprietà_funzione_probabilità} vale: \[
                0 < P(E_1 \cap \dotsm \cap E_{n-1}) \leq P(E_1 \cap \dotsm \cap E_{n-2}) \leq \dotsm \leq P(E_1)
            .\]
            Otteniamo quindi il seguente prodotto:
            \begin{align*}
                P(E_1 \cap \dotsm \cap E_n) &= P(E_1) \cdot \frac{P(E_1 \cap E_2)}{P(E_1)} \cdot \frac{P(E_1 \cap E_2 \cap E_3)}{P(E_1 \cap E_2)} \cdot \ldots \cdot \frac{P(E_1 \cap \dotsm \cap E_n)}{P(E_1 \cap \dotsm \cap E_{n-1})} \\ &= P(E_1) \cdot P(E_2|E_1) \cdot P(E_3|E_2 \cap E_1) \cdot \ldots \cdot P(E_n|E_1 \cap \dotsm \cap E_{n-1})
            .
            \end{align*}
            Si noti che abbiamo usato \eqref{eq:Formula_probabilità_condizionata} per riscrivere il prodotto precedente, ottenendo \eqref{eq:Formula_moltiplicazione}.
        \end{proof}
    \section{Eventi indipendenti}
        \begin{defn}\label{defn:Eventi_indipendenti}
            Sia $(\Omega,\,\mathscr{F},\,P)$ uno spazio di probabilità; gli eventi $E,\,F \in \mathscr{F}$ sono \textit{indipendenti} se vale: \[
                P(E \cup F) = P(E) \cdot P(F)
            .\] 
        \end{defn}
        \begin{obsv}
            Se due eventi $E$ ed $F$, presi dallo stesso spazio di probabilità, sono indipendenti allora valgono le seguenti uguaglianze:
            \begin{itemize}
                \item $P(E|F) = P(E)$;
                \item $P(F|E) = P(F)$.
            \end{itemize}
        \end{obsv}
        \begin{defn}
            Sia $(\Omega,\,\mathscr{F},\,P)$ uno spazio di probabilità; diciamo che gli eventi $E_1,\, \ldots,\, E_n$ sono \textit{indipendenti} se comunque preso un sottoinsieme $\{h_1,\, \ldots,\, h_k\} \subset \{1,\, \ldots,\, n\}$ con $k \geq 2$ vale la seguente uguaglianza: \[
                P(E_{h_1} \cap \dotsm \cap E_{h_k}) = P(E_{h_1}) \cdot \ldots \cdot P(E_{h_k})
            .\] 
        \end{defn}
        \begin{obsv}
            Per testare l'indipendenza di $m$ eventi, sarà necessario provare $2^{m} - m - 1$ uguaglianze (in dipendenza dalle possibili combinazioni di intersezioni tra gli eventi da testare).
        \end{obsv}
        \begin{defn}
            Preso uno spazio di probabilità e considerata una successione dei suoi eventi, diremo che essa è costituita da eventi indipendenti se, comunque scelto un sottoinsieme finito di eventi dalla successione, esso è costituito da eventi indipendenti.
        \end{defn}
        \begin{defn}
            Consideriamo uno spazio di probabilità $(\Omega,\,\mathscr{F},\,P)$, una successione di eventi $A_1,\, \ldots,\, A_n$ e un evento $F$ tale che $P(F) > 0$; gli eventi della successione si dicono \textit{condizionatamente indipendenti} dato $F$ se essi sono indipendenti rispetto alla probabilità $P(F)$.
        \end{defn}
    \section{Affidabilità dei sistemi}
        \begin{defn}
            La probabilità che un componente di un sistema non si guasti durante il periodo di tempo in cui deve operare è detta \textit{affidabilità} del componente; essa viene espressa rispetto all'interazione che i componenti hanno tra loro:
            \begin{itemize}
                \item \textbf{serie}: per garantire il funzionamento del sistema tutti i componenti collegati in serie devono funzionare correttamente; in questo caso vale \[
                    r = r_1 \cdot \ldots \cdot r_k
                ,\] per un sistema con $k$ componenti in serie;
            \item \textbf{parallelo}: per garantire il funzionamento del sistema basta che almeno un componente funzioni correttamente; in questo caso vale \[
                        r = 1 - (1 - r_1) \cdot \ldots \cdot (1 - r_k)
                ,\] per un sistema con $k$ componenti in parallelo.
            \end{itemize}
            In entrambi i casi appena mostrati, $r_i$ rappresenta l'affidabilità dell'$i$\nbdash esimo componente del sistema.
        \end{defn}
        \begin{obsv}
            Per analizzare problemi sull'affidabilità è conveniente scomporre un sistema complesso in sottosistemi che abbiano solo componenti in serie o solo in parallelo.
        \end{obsv}
    \section{Prove di Bernoulli}
        \begin{defn}
            Siano $n \in \mathbb{N}$ e $p \in (0,\,1)$; consideriamo il seguente spazio campionario: \[
                \Omega \coloneqq \{(a_1,\,\ldots,\,a_n)\,:\,a_k \in \{0,\,1\},\, k \in [1,\,n]\}
                ,\] la $\sigma$\nbdash algebra $\mathscr{F} = \mathscr{P}(\Omega)$ e la funzione di probabilità: \[
                \forall (a_1,\, \ldots,\, a_n) \in \Omega \,:\, P(\{a_1,\, \ldots,\, a_n\}) = p^{\left[\sum_{k=1}^{n} a_k\right]} \cdot (1 - p)^{\left[n - \sum_{k=1}^{n} a_k\right]}
            ;\] 
            la terna $(\Omega,\,\mathscr{F},\,P)$ si chiama \textit{spazio di probabilità di Bernoulli} oppure spazio di probabilità di $n$ prove di Bernoulli.
        \end{defn}
        \begin{prty}
            Consideriamo uno spazio di probabilità di Bernoulli nel quale la probabilità di successo della singola prova è $p \in (0,\,1)$; la probabilità di osservare $k \leq n$ successi in una sequenza di $n \geq 1$ prove di Bernoulli in questo spazio è data da:
            \begin{align}\label{eq:Probabilità_Bayes}
                \binom{n}{k} \cdot p^{k} \cdot (1 - p)^{n-k}
            .
            \end{align}
        \end{prty}
        \begin{proof}
            Sia $(\Omega,\,\mathscr{F},\,P )$ uno spazio di probabilità di $n$ prove Bernoulli e $B_k \in \mathscr{F}$ l'evento `osserviamo $k$ successi in $n$ prove', ovvero: \[
                B_k = \left\{ (a_1,\, \ldots,\, a_n) \in \Omega \,:\, \sum_{h=1}^{n} a_h = k \right\} 
            ;\] la probabilità di questo evento è definita come: \[
                P(B_k) = \sum_{\omega \in B_k} P(\{\omega\}) = \sum_{\omega \in B_k} p^{k}(1 - p)^{n - k} = |B_k|p^{k}(1 - p)^{n - k}
            .\] Osservando che $|B_k| = \binom{n}{k}$ (tutte le stringhe di $n$ bit contenenti $k$ zeri e $n - k$ uni) otteniamo l'equazione \eqref{eq:Probabilità_Bayes}
        \end{proof}
    \begin{obsv}
        Gli eventi $B_k$ che fissano $k$ successi in $n$ prove di Bernoulli hanno delle probabilità che corrispondono al modello binomiale $p_k$; possiamo affermare che, preso lo spazio campionario $\hat{\Omega} = \{0,\,\ldots,\,n\}$ dell'esperimento che considera $k$ successi in $n$ prove, lo spazio di probabilità di Bernoulli induce su $\hat{\Omega}$ un modello binomiale di parametri $n$ e $p$.
    \end{obsv} 
    \section{Serie geometrica}
    \begin{defn}[Serie]
        Data una successione $\{a_{n}\}$, se sommiamo gli infiniti termini otteniamo una \textit{serie}, definita come $\sum_{n=0}^{\infty} a_n$ (serie di a con n da 0 a infinito).

        Si dice somma parziale $n$\nbdash esima di una serie $s_{n}=\sum_{i=0}^{\infty} a_i\,n \in \mathbb{N}$ (la serie $\sum_{i=0}^{\infty} a_n$ converge, diverge o è irregolare se la successione delle sue somme parziali converge, diverge o è irregolare).

        Si dice somma di una serie il limite, se esiste finito, della successione delle sue somme parziali: $\lim_{n \to \infty} s_n = s$

        Condizione \underline{necessaria} (non sufficiente) affinché una serie converga è che il suo termine generale ($a_n$) tenda a 0
    \end{defn}
    \begin{defn}[Serie geometrica]\label{defn:Serie_geometrica}
        Chiamiamo \textit{geometrica} la serie col seguente termine generale:
        \begin{align*}
            \sum_{i=0}^{n} q^{i} \cdot \frac{1 - q^{n+1}}{1 - q} & &q\neq 1
        .\end{align*}
        La somma della serie con ragione $q \in (0,\,1)$ vale: \[
            \sum_{i=0}^{n} q^{i} = \begin{cases}
                \frac{1}{1-q} & \text{se $|q| < 1$}; \\
                +\infty & \text{se $q \geq 1$}; \\
                \text{indeterminata} & \text{altrimenti}
            \end{cases}
        .\] 
    \end{defn}

    %! TEX root = main.tex
% Capitolo 2

\chapter{Variabili Aleatorie}
    \section{Variabile aleatoria o casuale}
        \begin{defn}\label{defn:Variabile_aleatoria}
            Consideriamo lo spazio di probabilità $(\Omega,\,\mathscr{F},\,P)$; una \textit{variabile aleatoria} $X$ è una funzione da $\Omega$ in $\mathbb{R}$ tale che: \[
                \forall x \in \mathbb{R} \,:\, \{X \leq x\} \coloneqq \{\omega \in \Omega \,:\, X(\omega) \leq x\} \in \mathscr{F}
            .\]
        \end{defn}
        \begin{prty}\label{prty:Variabile_aleatoria}
            Se $X$ è una variabile aleatoria allora i seguenti insiemi sono eventi, ovvero sottoinsiemi di $\Omega$ che appartengono a $\mathscr{F}$:
            \begin{align*}
                &\{X < x\}, & &\{X \geq x\}, & &\{X > x\}, \\
                &\{x < X < y\}, & &\{x \leq X < y\}, & &\{x < X \leq y\}, \\
                &\{x \leq X \leq y\}, & &\{X = x\}, & &\{X \neq x\}
            .\end{align*}
        \end{prty}
    \section{Funzione di ripartizione}
        \begin{defn}\label{defn:Funzione_ripartizione}
            Sia $X$ una variabile aleatoria definita su uno spazio di probabilità $(\Omega,\,\mathscr{F},\,P)$; chiamiamo \textit{funzione di ripartizione} di $X$ la funzione $F_X\,:\, \mathbb{R} \mapsto [0,\,1]$ definita come: \[
                \forall x \in \mathbb{R} \,:\, F_X(x) \coloneqq P(X \leq x)
            .\]
        \end{defn}
        \begin{obsv}\label{obsv:Funzione_ripartizione}
            Sia $X$ una variabile aleatoria definita sullo spazio di probabilità $(\Omega,\,\mathbb{F},\,P)$ e sia dato $x \in \mathbb{R}$ che definisce la funzione di ripartizione $F_X(x) \coloneqq P(X \leq x)$; per il punto $(2)$ della Proprietà~\ref{prty:Spazio_di_probabilità} possiamo scrivere:
            \begin{enumerate}[\indent (a)]
                \item Consideriamo la probabilità inversa della funzione di ripartizione:
                    \begin{align*}
                        P(X > x) &= P(\{\omega \in \Omega \,:\, X(\omega) > x\}) = P(\{\omega \in \Omega \,:\, X(\omega) \leq x\}^{\text{C}}) \\
                                 &= 1 - P(\{\omega \in \Omega \,:\, X(\omega) \leq x\}) \\
                                 &= 1 - P(X \leq x)
                    .\end{align*}
                \item Se invece abbiamo $x,\,y \in \mathbb{R}$ con $x < y$, per il punto $(3)$ della Proprietà~\ref{prty:Proprietà_funzione_probabilità} vale:
                    \begin{align*}
                        P(x < X \leq y) &= P(\{\omega \in \Omega \,:\, x < X(\omega) \leq y\}) \\
                                     &= P(\{\omega \in \Omega \,:\, X(\omega) \leq y\} \backslash \{\omega \in \Omega \,:\, X(\omega) \leq x\}) \\
                                     &= P(\{\omega \in \Omega \,:\, X(\omega) \leq y\} - \{\omega \in \Omega \,:\, X(\omega) \leq x\}) \\
                                     &= P(X \leq y) - P(X \leq x)
                    .\end{align*}
            \end{enumerate}
            Osserviamo che la conoscenza della funzione di ripartizione di $X$ ci permette di calcolare le probabilità di eventi ad essa associati.
            \begin{prty}\label{prty:Funzione_ripartizione}
                Sia $X$ una variabile aleatoria definita sullo spazio di probabilità $(\Omega,\,\mathbb{F},\,P)$ e la sua funzione di ripartizione $F_X(x) = P(X \leq x)$; allora possiamo affermare che:
                \begin{itemize}
                    \item $F_X(x)$ è una funzione \textit{monotona non decrescente};
                    \item $F_X(x)$ è \textit{continua} da destra: $\forall x_0 \in \mathbb{R} \,:\, \lim_{x \downarrow x_0} F_X(x) = F_X(x_0)$;
                    \item il limite destro ($x \rightarrow +\infty$) di $F_X(x)$ vale 1, il limite sinistro ($x \rightarrow -\infty$) vale 0.
                \end{itemize}
            \end{prty}
            \begin{obsv}
                Si può dimostrare che, data una funzione $F$ che soddisfa le Proprietà~\ref{prty:Funzione_ripartizione}, esiste uno spazio di probabilità $(\Omega,\,\mathbb{F},\,P)$ e una variabile aleatoria $X$ definita su di esso che abbia $F$ come funzione di ripartizione; possiamo trattare il problema della A variabile aleatoria $X$ con funzione di ripartizione $F$ senza dover costruire lo spazio di probabilità che contiene $X$.
            \end{obsv}
        \end{obsv}
    \section{Variabili aleatorie discrete}
    \begin{defn}\label{defn:Variabili_aleatorie_discrete}
            La variabile aleatoria $X$ definita su uno spazio di probabilità $(\Omega,\,\mathbb{F},\,P)$ è una variabile aleatoria \textit{discreta} se assume con probabilità 1 i valori all'interno di un insieme $S$ al più numerabile: \[
                P(X \in S) = 1
            .\] 
        \end{defn}
        \begin{defn}\label{defn:Densità_discreta}
            Sia $X$ una variabile aleatoria \underline{discreta} su uno spazio di probabilità $(\Omega,\,\mathbb{F},\,P)$; allora chiamiamo \textit{densità discreta} di $X$ la funzione definita come segue: \[
                p_X(x) \coloneqq P(X = x)
            .\]
        \end{defn}
        \begin{prty}\label{prty:Densità_discreta}
            Sia  $p_X(x)$ la densità della variabile aleatoria discreta $X$, che assume con probabilità 1 i valori in $S = \{x_k \,:\, k \in I \subset \mathbb{Z}\}$; allora possiamo affermare che:
            \begin{enumerate}
                \item $\forall x \notin S \,:\, p_X(x) = 0 \;\land\; \forall x \in S \,:\, 0 \leq p_X(x) \leq 1$;
                \item $\sum_{k \in I} p_X(x_k) = 1$;
                \item se la funzione di ripartizione di $X$ è $F_X(x)$ allora vale:  \[
                    \forall x \in \mathbb{R} \,:\, F_X(x) = \sum_{k\,:\,x_k \leq x} p_X(x_k)
                ;\]
                \item se possiamo numerare i punti di S prendendo $x_k,\,x_h \in S \,:\, h < k \implies x_h < x_k$ allora vale: \[
                    \forall k \in I \,:\, p_X(x_k) = F_X(x_k) - F_X(x_{k-1})
                ;\]
                \item se consideriamo un sottoinsieme $B \subset \mathbb{R}$ vale: \[
                    P(X \in B) = \sum_{k \,:\, x_k \in B} p_X(x_k)
                .\]
            \end{enumerate}
        \end{prty}
        \begin{proof}
            \hfill
            \begin{enumerate}
                \item Dalla Definizione~\ref{defn:Densità_discreta} segue direttamente che, data la probabilità $p_X(x) = P(X = x)$ e sapendo che $x$ assume valori in un insieme numerabile $S$, otteniamo la prova di $(1)$.
                \item abbiamo enunciato l'ipotesi per cui $P(X \in S) = 1$; possiamo scrivere  \[
                        1 = P(X \in S) = P\left( \bigcup_{k \in I} {X = x_k} \right) = \sum_{k \in I} P(X = x_k) = \sum_{k \in I} p_X(x_k)
                .\]
                \item Osservando che la funzione di ripartizione di $X$ vale $F_X(x) = P(X \leq x)$ e ricordando l'ipotesi $P(X \in S) = 1$ otteniamo:
                \begin{align*}
                    F_X(x) &= P(X \leq x) = P(X \in (-\infty,\,x] \cap S) \\
                           &= P\left( \bigcup_{k \,:\, x_k \leq x} {X = x_k}\right) = \sum_{k \,:\, x_k \leq x} P(X \leq x_k) = \sum_{k \,:\, x_k \leq x} p_X(x_k)
                .\end{align*}
                \item Per il punto (b) dell'Osservazione~\ref{obsv:Funzione_ripartizione} segue che: \[
                    F_X(x_k) - F_X(x_{k-1}) = P(x_{k-1} < X \leq x_k)
                ;\] se i punti di $S$ sono numerati come nell'ipotesi $h < k \implies x_h < x_k$ allora otteniamo: \[
                    P(x_{k-1} < X \leq x_k) = P(X = x_k) \coloneqq p_X(x_k)
                .\]
            \item Riprendiamo l'ipotesi $P(X \in S) = 1$:
                \begin{align*}
                    P(X \in B) &= P(X \in B \cap S) \\
                               &= P\left(\bigcup_{k\,:\,x_k \in B \cap S} \{X = x_k\}\right) = \sum_{k\,:\,x_k \in B \cap S} P(X = x_k) = \sum_{k\,:\,x_k \in B \cap S} p_X(x_k)
                .\qedhere
                \end{align*}
            \end{enumerate}
        \end{proof}
        \begin{obsv}
            Sia $S = \{x_k\,:\,k \in I \subset \mathbb{Z}\} \subset \mathbb{R}$; una funzione $p\,:\, \mathbb{R} \mapsto \mathbb{R}$ è una densità discreta su $S$ se essa soddisfa gli assiomi $(1)$ e $(2)$  della Proprietà~\ref{prty:Densità_discreta}.
        \end{obsv}
    \section{Densità notevoli}
        \subsection{Binomiale}
            \begin{defn}\label{defn:Densità_Binomiale}
                Definiamo la variabile aleatoria $X$ come il numero di successi ottenuti in $n$ prove di Bernoulli; tale variabile aleatoria potrà assumere solamente i valori $0,\, \ldots,\, n$ (è una variabile aleatoria discreta).
                La sua densità vale:
                \begin{equation}\label{eq:Densità_Binomiale}
                    p_X(k) = P(X = k) = \begin{cases}
                        \binom{n}{k}\cdot p^k\cdot (1-p)^{n-k} & \text{se $k \in {0,\, \ldots,\, n}$;}\\
                        0 & \text{se $k \notin {0,\, \ldots,\, n}$.}
                    \end{cases}
                \end{equation}
                Chiamiamo la \eqref{eq:Densità_Binomiale} \textit{densità binomiale di parametri $n$ e $p$} o in modo equivalente indichiamo: \[
                    X \sim \mathcal{B}i(n,\,p)
                .\]
            \end{defn}
            \begin{defn}\label{defn:Densità_Bernoulliana}
                Consideriamo una variabile aleatoria con densità binomiale $X \sim \mathcal{B}i(n,\,p)$ dove $n = 1$; questa variabile rappresenta il numero di successi in una sola prova, ovvero $X \in {0,\, 1}$ e la sua densità vale:
                \begin{equation}\label{eq:Densità_Bernoulliana}
                    \begin{cases}
                        1-p & \text{se $k=0$;} \\
                        p & \text{se $k=1$;} \\
                        0 & \text{se $k \notin {0,\, 1}$.}
                    \end{cases}
                \end{equation}
                Chiamiamo la \eqref{eq:Densità_Binomiale} \textit{densità bernoulliana di parametro $p$} o in modo equivalente indichiamo: \[
                    X \sim \mathcal{B}e(p)
                .\] 
            \end{defn}
            \begin{obsv}
                La variabile aleatoria costante di valore 1 si indica come una variabile aleatoria con densità bernoulliana di parametro 0 ($X \equiv 0 \implies X \sim \mathcal{B}e(0)$); la variabile aleatoria costante di valore 0 si indica come una variabile aleatoria con densità bernoulliana di parametro 1 ($X \equiv 1 \implies X \sim \mathcal{B}e(1)$).
            \end{obsv}
        \subsection{Geometrica}
            \begin{defn}
                Definiamo la variabile aleatoria $X$ l'istante di tempo discreto in cui si verifica un evento, e tale evento sia indipendente dal tempo al quale si verifica; fissato un istante di tempo $k$ possiamo scrivere: \[
                    \forall k \in \mathbb{N} \,:\, P(X > k + 1 | X > k) = P(X > 1)
                .\] Se conosciamo la probabilità $q \coloneqq P(X > 1)$ possiamo ottenere la densità di $X$; usiamo \eqref{eq:Formula_probabilità_condizionata} per riscrivere $q$: \[
                    \forall k \in \mathbb{N} \,:\, q = P(X > 1) = \frac{P(X > k + 1 \cap X > k)}{P(X > k)} = \frac{P(X > k + 1)}{P(X > k)}
                .\] Ora invertiamo la precedente in questo modo:
                \begin{align*}
                    \forall k \in \mathbb{N} \,:\, P(X > k + 1) &= P(X > 1) \cdot P(X > k) \\
                                                        &= P(X > 1)^{k+1} = q^{k+1}
                .\end{align*}
                Sapendo che la funzione di ripartizione di $X$ vale $F_X(k) = 1 - P(X > k) = 1 - q^k$, possiamo ottenere la densità usando il punto $(4)$ della Proprietà~\ref{prty:Densità_discreta}: \[
                    P(X = k) = F_X(k) - F_X(k-1) = q^{k-1} - q^k = q^{k-1}(1-q)
                .\] Definendo l'\textit{intensità di guasto} come la quantità $p \coloneqq 1-q = P(X \leq 1)$ otteniamo la seguente scrittura della densità:
                \begin{align*}
                    P(X = k) = p \cdot (1-p)^{k-1} & &\forall k \in \mathbb{N}
                .\end{align*}
            \end{defn}
            \begin{obsv}
                Richiamiamo la Definizione~\ref{defn:Serie_geometrica}, e dalla somma della serie geometrica otteniamo: \[
                    P(X \in \mathbb{N}) = \sum_{k=1}^{\infty} P(X = k) = \sum_{k=1}^{\infty} p \cdot (1-p)^{k-1} = p \cdot \sum_{k=0}^{\infty} (1-p)^k = p \cdot \frac{1}{1-(1-p)} = 1
                .\] 
            \end{obsv}
            \begin{defn}
                Dalla precedente osservazione e dalla Definizione~\ref{defn:Densità_discreta} concludiamo che $X$ è una variabile aleatoria con la seguente densità:
                \begin{equation}\label{eq:Densità_Geometrica}
                    p_X(k) = \begin{cases}
                        p \cdot (1-p)^{k-1} & \text{se $k \in \mathbb{N}$;} \\
                        0 & \text{se $k \notin \mathbb{N}$.}
                    \end{cases}
                \end{equation}
                Chiamiamo la \eqref{eq:Densità_Geometrica} \textit{densità geometrica di parametro $p$} o in modo equivalente indichiamo: \[
                    X \sim \mathcal{G}(p)
                .\] 
            \end{defn}
        \subsection{Poisson}
            \begin{defn}
                Consideriamo una variabile aleatoria $X \sim \mathcal{B}i(n,\,p)$ con parametro $n \gg 1$ e parametro  $p \ll 1 \land p \in (0,\, 1)$; definiamo inoltre $\lambda \coloneqq n \cdot p$, il quale non dovrà risultare eccessivamente grande (rispetto a $n$).

                Nelle ipotesi appena dettate, perché con un numero grande $n$ di prove indipendenti, aventi ciascuna probabilità di successo $p$ bassa, si abbia un successo, una condizione fondamentale è il valore di $\lambda$, legato alla densità della variabile aleatoria nel modo seguente: \[
                    X \sim \mathcal{B}i(n,\, \lambda / n)
                .\] La probabilità sottostante si scrive come:
                \begin{align*}
                    P(X = x) &= \binom{n}{k} \left(\frac{\lambda}{n}\right)^k \left(1-\frac{\lambda}{n}\right)^{n-k} \\
                             &= \frac{n!}{(n-k)!k!} \cdot \left(\frac{\lambda}{n}\right)^k \cdot \left(1-\frac{\lambda}{n}\right)^{n-k} \\
                             &= \underset{(1)}{\underbrace{\frac{n!}{(n-k)!n^k}}} \cdot \underset{(2)}{\underbrace{\left(1-\frac{\lambda}{n}\right)^{-k}}} \cdot \frac{\lambda^k}{k!}\underset{(3)}{\underbrace{\left(1-\frac{\lambda}{n}\right)^{n}}}
                .\end{align*}
                Analizziamo i tre termini evidenziati, al limite per $n$ grande:
                \begin{enumerate}
                    \item $\lim_{n \to \infty} \frac{n!}{(n-k)!n^k} = 1$;
                    \item $\lim_{n \to \infty} \left(1-\frac{\lambda}{n}\right)^{-k} = 1$;
                    \item $\lim_{n \to \infty} \left(1-\frac{\lambda}{n}\right)^{n} = e^{-\lambda}$.
                \end{enumerate}
                Concludiamo che la densità può essere approssimata nel modo seguente, sotto le nostre ipotesi:
                \begin{align}\label{eq:Approssimazione_Binomiale_Poisson}
                        P(X = k) \simeq \frac{\lambda^k}{k!} \cdot e^{-\lambda} & &k \in \mathbb{N},\, \lambda = n \cdot p
                    .
                \end{align}

                Per $\lambda > 0$ possiamo dire che $X$ è una variabile aleatoria con la seguente densità:
                \begin{equation}\label{eq:Densità_Poisson}
                    p_X(k) = \begin{cases}
                        \frac{e^{-k}\lambda^k}{k!} & \text{se $k \in \mathbb{N}$;} \\
                        0 & \text{se $k \notin \mathbb{N}$.}
                    \end{cases}
                \end{equation}
                Chiamiamo la \eqref{eq:Densità_Poisson} \textit{densità di Poisson di parametro $\lambda$} o in modo equivalente indichiamo: \[
                    X \sim \mathcal{P}(\lambda)
                .\] 
            \end{defn}
            \begin{obsv}
                Possiamo usare \eqref{eq:Approssimazione_Binomiale_Poisson} per il calcolo approssimato di $P(X = k)$ quando abbiamo:
                \begin{itemize}
                    \item $X \sim \mathcal{B}i(n,\,p)$;
                    \item $n \gg 1$;
                    \item $p \ll 1$.
                \end{itemize}
                In questo modo si evita il calcolo dei coefficienti binomiali per ciascun $k$.
            \end{obsv}
        \subsection{Ipergeometrica}
            \begin{defn}
                Consideriamo una variabile aleatoria $X$ che conta in numero di successi in $n$ prove, con probabilità di successo determinata da $r$ esiti favorevoli e $b$ esiti sfavorevoli; inoltre, ogni prova determina l'estrazione dallo spazio campionario di un $r$ o di un $b$.
                Ricaviamo la seguente considerazione: \[
                    X < \scriptstyle\text{MIN}\textstyle(n,\,r) \land X > \scriptstyle\text{MAX}\textstyle(0,\,n-b) \implies X \in S = \{\scriptstyle\text{MAX}\textstyle(0,\,n-b),\, \scriptstyle\text{MAX}\textstyle(0,\,n-b) + 1,\, \ldots,\, \scriptstyle\text{MIN}\textstyle(n,\,r)\}
                .\] Scegliendo un $k \in S$ possiamo calcolare la densità $P(X = k)$ come casi favorevoli su casi possibili:
                \begin{equation}\label{eq:Densità_Ipergeometrica}
                    p_X(k) = P(X = k) = \begin{cases}
                        \frac{\binom{r}{k} \cdot \binom{b}{n-k}}{\binom{r+b}{n}} & \text{se $k \in S$;} \\
                        0 & \text{se $k \notin S$.}
                    \end{cases}
                \end{equation}
                Chiamiamo la \eqref{eq:Densità_Ipergeometrica} \textit{densità ipergeometrica di parametri $b+r$, $r$ e $n$} o in modo equivalente indichiamo: \[
                    X \sim \mathcal{I}(b+r,\, r,\, n)
                .\] 
            \end{defn}
            \begin{obsv}
                Se il valore $r+b$ è molto grande (tende a $+\infty$) allora la dipendenza tra le prove successive si attenua e, in opportune condizioni, è possibile approssimare una variabile aleatoria $X$ con densità ipergeometrica tramite la densità binomiale: \[
                    X \sim \mathcal{I}(b+r,\, r,\, n) \simeq \mathcal{B}i(n,\, r / (r+b))
                .\] 
            \end{obsv}
    \section{Variabili aleatorie continue}
        \begin{defn}\label{defn:Variabile_aleatoria_continua}
            Consideriamo una variabile aleatoria $X$ definita su uno spazio di probabilità $(\Omega,\,\mathbb{F},\,P)$; essa è chiamata variabile aleatoria \textit{assolutamente continua} se esiste una funzione $f_X\,:\, \mathbb{R} \mapsto \mathbb{R^+}$, che sia integrabile e la primitiva della funzione di ripartizione di $X$; inoltre deve essere possibile scrivere $F_X(x)$ nel modo seguente:
            \begin{equation}
                F_X(x) = \int_{-\infty}^{x} f_X(s)\, ds 
            \end{equation}
            Chiamiamo \textit{densità} di $X$ la funzione $f_X(x)$.
        \end{defn}
        \begin{prty}\label{prty:Variabile_aleatoria_continua}
            Se $f_x(x)$ è la densità di una variabile aleatoria $X$ assolutamente continua, allora possiamo affermare che:
            \begin{enumerate}
                \item $\int_{\mathbb{R}} f_X(x)\, dx = 1$;
                \item se risulta che la primitiva di $f_X(x)$ è la funzione di ripartizione di $X$, allora vale: \[
                        \forall x \in \mathbb{R} \,:\, \exists F^{\prime}_X(x) \implies f_X(x) = F_X^{\prime}(x)
                ;\]
            \item se presi $a,\, b \in \mathbb{R}$ si verifica che $-\infty < a < b < +\infty$, allora vale:
                \begin{align*}
                    P\left(X \in (a,\,b)\right) &= P\left(X \in (a,\,b]\right) = P\left(X \in [a,\,b)\right) = P\left(X \in [a,\,b]\right) \\
                                   &= \int_{a}^{b} f_X(x)\, dx
                .\end{align*}
            \end{enumerate}
            \begin{proof}
                \hfill
                \begin{enumerate}
                    \item $1 = \lim_{x \to +\infty} F_X(x) = \lim_{x \to +\infty} \left(\int_{-\infty}^{x} f_X(s)\, ds\right) = \int_{\mathbb{R}} f_X(s)\, ds$;
                    \item conseguenza del teorema fondamentale del calcolo integrale;
                    \item sapendo che $\forall x \in \mathbb{R} \,:\, P(X = x) = 0$ otteniamo: \[
                        P(X \in (a,\,b]) = P({X \in (a,\,b)} \cup {X = b}) = P(X \in (a,\,b)) + \cancel{P(X = b)} = P(X \in (a,\,b))
                    .\] In modo analogo possiamo dimostrare che vale:\[
                        P(X \in (a\,b)) = P(X \in [a\,b)) = P(X \in [a\,b]) 
                    .\] Consideriamo l'intervallo $(a\,b]$:
                    \begin{align*}
                        P(X \in (a,\,b]) &= P(\{X \in (-\infty,\,b]\} \backslash \{X \in (-\infty,\,a]\}) \\
                                      &= P(\{X \in (-\infty,\,b]\} - \{X \in (-\infty,\,a]\}) \\
                                      &= F_X(b) - F_X(a) = \int_{-\infty}^{b} f_X(x)\, dx - \int_{-\infty}^{a} f_X(x)\, dx \\
                                      &= \int_{a}^{b} f_X(x)\, dx \qedhere
                    .\end{align*}
                \end{enumerate}
            \end{proof}
            \begin{prty}
                Consideriamo il punto $(2)$ della Proprietà~\ref{prty:Variabile_aleatoria_continua}.

                Sia $X$ una variabile aleatoria e $F_X(x)$ la sua funzione di ripartizione; se $F_X(x)$ è continua e derivabile con continuità per ogni $x$ in $\mathbb{R}$ \--- eccetto al più un insieme finito di punti $B \coloneqq \{x_1,\,\ldots,\, x_n\} \subset \mathbb{R}$; allora $X$ è una variabile aleatoria \textit{assolutamente continua} e la funzione $f_X(x) = F^{\prime}_X(x)$, definita $\forall x \notin B$ in modo arbitrario su $B$, è una densità per $X$.
            \end{prty}
            \begin{obsv}
                Consideriamo il punto $(3)$ della Proprietà~\ref{prty:Variabile_aleatoria_continua}.

                Sia $X$ una variabile aleatoria \underline{assolutamente continua} con densità $f_X(x)$ e l'insieme $B \subset \mathbb{R}$ sia tale che $B = B_1 \cup B_2 \cup \dotsm$ dove gli intervalli $B_k,\, k = 0,\,1,\,\ldots$ sono disgiunti; allora otteniamo: \[
                    P(X \in B) = \int_{B} f_X(x)\, dx = \sum_{k=1}^{+\infty} \int_{B_k} f_X(x)\, dx
                .\] 
            \end{obsv}
            \begin{obsv}
                Una funzione $f\,:\, \mathbb{R} \mapsto \mathbb{R}$ è una \textit{densità} su $\mathbb{R}$ se valgono:
                \begin{enumerate}
                    \item $f(x)$ è integrabile e $f(x) \geq 0$, per ogni $x$ in $\mathbb{R}$;
                    \item $\int_{\mathbb{R}} f(x)\, dx = 1$.
                \end{enumerate}
            \end{obsv}
        \end{prty}
    \section{Densità continue notevoli}
        \subsection{Bernoulli}
        \subsection{Binomiale}
        \subsection{Geometrica}
        \subsection{Poisson}
        \subsection{Ipergeometrica}
        \subsection{Uniforme}
        \subsection{Esponenziale}
        \subsection{Normale}
        \subsection{Normale Standard}
        \subsection*{Assenza di memoria di geometrica ed esponenziale}
    \section{Struttura della distribuzione geometrica}
    \section{Valore atteso e varianza}
    \section{Deviazione standard}
    \section{Funzioni di variabili aleatorie}
    \section{Trasformazioni affini di variabili aleatorie}
    \section{Standardizzazione di variabile aleatoria}
    \section{Probabilità della normale standard}
    \section{Disuguaglianza di Chebichev}
    \section{Teorema di De Moivre-Laplace}
        \subsection{Approssimazione normale della binomiale}
        \subsection*{Correzione del continuo}
    \section{Momenti}
    \section{Affidabilità}
        \subsection{Tempi di vita}
        \subsection{Intensità di guasto}
        \subsection{Distribuzione di Weibull}

    %! TEX root = main.tex
% Capitolo 3

\chapter{Vettori Aleatori}
    \section{Vettore aleatorio}
        \begin{defn}
            Consideriamo $n$ variabili aleatorie $X_1, \ldots,\, X_n$ definite tutte sullo stesso spazio di probabilità $(\Omega,\,\mathbb{F},\,P)$; diremo che sono  \textit{indipendenti} se vale:
            \begin{equation}\label{eq:Indipendenza_variabili_aleatorie}
                P(X_1 \in B_1,\, \ldots,\, X_n \in B_n) = P(X_1 \in B_1) \cdot \ldots \cdot P(X_n \in B_n)
            ,\end{equation}
            per ogni scelta dei domini regolari $B_1,\, \ldots,\, B_n$ ottenuti al più con un numero finito e numerabile di operazioni tra intervalli.
        \end{defn}
        \begin{obsv}
            Se prendiamo dei domini regolari definiti come: \[
                \forall i \in (-\infty,\, \ldots x_i]
            ,\] l'equazione \eqref{eq:Indipendenza_variabili_aleatorie} diventa:
            \begin{equation}\label{eq:Indipendenza_v_a_ripartizione}
                P(X_1 \leq x\,\, \ldots,\, X_N \leq x_n) = P(X_1 \leq x_1) \cdot \ldots \cdot P(X_n \leq x_n)
            .\end{equation}
        \end{obsv}
        \begin{prty}
            Prese $n$ variabili aleatorie $X_1,\, \ldots,\, X_n$, esse sono indipendenti se vale  \eqref{eq:Indipendenza_v_a_ripartizione} per ogni scelta di $x_1,\, \ldots,\, x_n \in \mathbb{R}$.
        \end{prty}
        \begin{obsv}
            Consideriamo $n$ variabili aleatorie indipendenti e discrete $X_1,\, \ldots,\, X_n$ con densità rispettivamente $p_{X_1},\, \ldots,\, p_{X_n}$; allora scegliendo i domini in \eqref{eq:Indipendenza_variabili_aleatorie} come: \[
                B_1 = \{x_1\},\, \ldots,\, B_n = \{x_n\}
            ,\] otteniamo che:
            \begin{align}\label{eq:Indipendenza_v_a_densità}
                P(X\ = x_1,\, \ldots,\, X_n = x_n) = P(X_1 = x_1) \cdot \ldots \cdot P(X_n = x_n) & &
                \forall  x_i \in \mathbb{R} \forall i = 1,\, \ldots,\, n
            .\end{align}
        \end{obsv}
        \begin{prty}
            Possiamo affermare che, prese $n$ variabili aleatorie discrete $X_1,\, \ldots,\, X_n$, esse sono indipendenti se vale \eqref{eq:Indipendenza_v_a_densità}.
        \end{prty}
    \section{Vettori aleatori}
        \begin{defn}
            Sia dato uno spazio di probabilità $(\Omega,\,\mathbb{F},\,P)$; un \textit{vettore aleatorio} $n$\nbdash dimensionale è una funzione vettoriale definita come:
            \begin{align*}
                \vec{X} \coloneqq (X_1,\, \ldots,\, X_{n}) & &
                \vec{X}\,:\, \Omega \mapsto \mathbb{R}^n
            ,\end{align*}
            tale che per ogni $i \in 1,\, \ldots,\, n$ ciascun $X_i$ sia una variabile aleatoria.
        \end{defn}
    \section{Funzione di ripartizione congiunta}
        \begin{defn}
            Sia $\vec{X} = (X_1,\, \ldots,\, X_{n})$ un vettore aleatorio $n$\nbdash dimensionale definito sullo spazio di probabilità $(\Omega,\,\mathbb{F},\,P)$; chiamiamo funzione di ripartizione di $\vec{X}$ (oppure funzione di ripartizione \textit{congiunta} di $X_1,\, \ldots,\, X_{n}$) la funzione: \[
                F_{\vec{X}} = F_{(X_1,\, \ldots,\, X_{n})} \,:\, \mathbb{R}^n \mapsto [0,\,1]
            ,\] definita $\forall x \in (x_1,\, \ldots,\, x_{n}) \in \mathbb{R}^n$ come: \[
                F_{(X_1,\, \ldots,\, X_{n})}(x_1,\, \ldots,\, x_{n}) \coloneqq P(X_1 \leq x_1,\, \ldots,\, X_n \leq x_n)
            .\]
        \end{defn}
        \begin{prty}\label{prty:Convergenza_ripartizione_congiunta}
            Sia $\vec{X} = (X_1,\, \ldots,\, X_{n})$ un vettore aleatorio che ammette funzione di ripartizione $F_{\vec{X}}$ e sia $\vec{x} = (x_1,\, \ldots,\, x_{n})$; allora vale: \[
                \forall k \in [1,\,n] \,:\, \lim_{x_k \to \infty} F_{\vec{X}}(\vec{x}) = 0
            ,\] mentre otteniamo:
            \begin{align*}
                \lim_{x_k \to \infty} F_{\vec{X}}(\vec{x}) &= 
                P(X_1 \leq x_1,\, \ldots,\, X_{k-1} \leq x_{k-1},\, X_{k+1} \leq x_{k+1},\, \ldots,\, X_n \leq x_n) \\
                                                           &= F_{(X_1,\, \ldots,\, X_{k-1},\, X_{k+1},\, \ldots,\, X_{n})}(x_1,\, \ldots,\, x_{k-1},\, x_{k+1},\, \ldots,\, x_{n})
            .\end{align*}
        \end{prty}
    \section{Indipendenza di variabili aleatorie}
        \begin{obsv}
            Consideriamo un vettore aleatorio bidimensionale $(X,\,Y)$ con funzione di ripartizione $F_{X,\,Y}$; la Proprietà~\ref{prty:Convergenza_ripartizione_congiunta} afferma che:
            \begin{gather*}
                \lim_{x \to \infty} F_{X,\,Y}(x,\,y) = P(Y \leq y) = F_Y(y); \\
                \lim_{y \to \infty} F_{X,\,Y}(x,\,y) = P(X \leq y = F_X(x)
            .\end{gather*}
            Se prendiamo in generale un vettore aleatorio $n$\nbdash dimensionale, applicando la Proprietà~\ref{prty:Convergenza_ripartizione_congiunta} iterativamente, otteniamo:
            \begin{align*}
                F_{X_i}(x) = \lim_{\substack{x_j \rightarrow +\infty \\ \forall j \neq i}} F_{\vec{X}}(x_1,\, \ldots,\, x_{i-1},\, x,\, x_{i+1},\, \ldots,\, x_{n})
            .\end{align*}
            Concludiamo che dalla funzione di ripartizione congiunta si possono calcolare le ripartizioni marginali, ma non vale il contrario.
        \end{obsv}
        \begin{defn}
            Le componenti di un vettore aleatorio $\vec{X} = (X_1,\, \ldots,\, X_{n})$ sono indipendenti se e solo se la funzione di ripartizione di $\vec{X}$ coincide col prodotto delle ripartizioni marginali: \[
            F_{\vec{X}} = F_{X_1} \cdot \ldots \cdot F_{X_n}
            .\] 
        \end{defn}
    \section{Vettori aleatori discreti}
        \begin{defn}
            Un vettore aleatorio $n$\nbdash dimensionale $X$ è \textit{discreto} se le sue componenti $X_1,\, \ldots,\, X_{n}$ sono variabili aleatorie discrete.
        \end{defn}
    \section{Densità discrete}
        \begin{defn}
            Sia $\vec{X}$ un vettore aleatorio discreto su uno spazio di probabilità $(\Omega,\,\mathbb{F},\,P)$; la funzione: \[
                p_{\vec{X}} \coloneqq P(X_1 = x_1,\, \ldots,\, X_n = x_n)
            ,\] dove vale $\vec{x} = (x_1,\, \ldots,\, x_{n})$, si chiama  \textit{densità discreta} del vettore aleatorio $\vec{X}$ oppure densità congiunta di $X_1,\, \ldots,\, X_{n}$.
        \end{defn}
    \section{Distribuzione multinomiale}
        \begin{defn}
            
        \end{defn}
    \section{Variabili aleatorie congiuntamente continue}
    \section{Densità di probabilità congiunta}
    \section{Funzioni di vettori aleatori}
    \section{Trasformazioni affini di vettori aleatori}
    \section{Convoluzione discreta e continua}
    \section{Valore atteso per funzioni di vettori aleatori}
    \section{Proprietà del valore atteso per due vettori aleatori}
    \section{Valore atteso del prodotto di due variabili aleatorie}
    \section{Varianza della somma di due variabili aleatorie}
    \section{Covarianza}
    \section{Media e varianza della Binomiale}
    \section{Coefficiente di correlazione lineare}
    \section{Somme di variabili aleatorie indipendenti}
    \section{Distribuzione del massimo e del minimo}
    \section{Media campionaria}
    \section{Legge dei grandi numeri}
    \section{Teorema del limite centrale} 

    %! TEX root = main.tex
% Capitolo 4

\chapter{Distribuzioni notevoli}
    \section{Matrice delle covarianza}
    \section{Gaussiane multivariate}
    \section{Generatrice dei momenti}
    \section{Somma esponenziali indipendenti}
    \section{Distribuzione Gamma}
    \section{Quantili}
    \section{Distribuzioni derivate dalla Normale}

    \part{Statistica inferenziale}
    %! TEX root = main.tex
% Capitolo 5

\chapter{Modelli statistici e stima}
    \section{Campione aleatorio}
        \begin{defn}[Campione e Inferenza]
            Un insieme $X_1,\, \ldots,\, X_{n}$ di $n$ variabili aleatorie indipendenti, in cui ciascuna 
            variabile aleatoria abbia stessa distribuzione $F$, è detto \emph{campione} della 
            distribuzione $F$.

            Non ostante la distribuzione $F$ non sia nota, possiamo usare i dati ricavabili dal campione
            per fare dell'\emph{inferenza} su $F$. Se essa è nota a meno di un insieme di parametri 
            incogniti, abbiamo un problema di inferenza \emph{parametrica}; nel caso in cui non si conosce 
            nulla di $F$, si ha un problema di inferenza \emph{non parametrica}.
        \end{defn}
        \begin{defn}[Realizzazione]
            Se dato un campione $X_1,\, \ldots,\, X_{n}$ di una distribuzione $F$ osserviamo i valori 
            $X_1=x_1,\, \ldots,\, X_{n}=x_n$, chiamiamo il vettore $(x_1,\, \ldots,\, x_n)$ 
            \emph{realizzazione} del campione (dati).
        \end{defn}
    \section{Statistica}
        \begin{defn}
            Consideriamo un campione aleatorio $X_1,\, \ldots,\, X_{n}$ estratto da $F$; una 
            \emph{statistica} basata sul campione è una funzione nota del campione: \[
                D_n = d_n(X_1,\, \ldots,\, X_{n})
            ;\] la statistica $D_n$ è una variabile aleatoria.
        \end{defn}
    \section{Caratteristica della popolazione}
        \begin{defn}
            Consideriamo un campione $X_1,\, \ldots,\, X_{n}$ estratto da una popolazione con 
            distribuzione $F_\vartheta$, nota a meno di un parametro $\vartheta$ incognito; allora 
            chiamiamo \emph{caratteristica} della popolazione una funzione non costante di $\vartheta$.
        \end{defn}
    \section{Stimatore e stima}
        \begin{defn}[Stimatore]
            Sia $X_1,\, \ldots,\, X_{n}$ un campione aleatorio estratto da $F_\vartheta$, e $k(\vartheta)$ 
            una caratteristica della popolazione $F_\vartheta$; allora chiamiamo \emph{stimatore} 
            di $k(\vartheta)$ una \underline{statistica} \[
                \hat{K}_n = d_n(X_1,\, \ldots,\, X_{n})
            ,\] usata per fare inferenza su $k(\vartheta)$.
        \end{defn}
        \begin{defn}[Stima]
            Data la \underline{realizzazione} $(x_1,\, \ldots,\, x_{n})$ e lo stimatore 
            $\hat{K}_n = d_n(X_1,\, \ldots,\, X_{n})$ della caratteristica $k(\vartheta)$,
            chiamiamo \emph{stima} il valore della statistica $\hat{K}_n$ in corrispondenza 
            della realizzazione $(x_1,\, \ldots,\, x_{n})$: \[
                \hat{k}_n = d_n(x_1,\, \ldots,\, x_{n})
            ;\] mentre lo stimatore è una statistica (quindi una variabile aleatoria), 
            la stima è un valore numerico.
        \end{defn}
    \section{Errore quadratico medio}
    \begin{defn}
        Sia dato uno stimatore $D_n = d_n(X_1,\, \ldots,\, X_{n})$ di $k(\vartheta)$ che ammette 
        momento secondo finito; allora chiamiamo \emph{errore quadratico medio} di $D_n$ 
        la seguente funzione di $\vartheta$ definita positiva:
        \begin{equation}\label{eq:Errore_quadratico_medio}
            r_{\vartheta}\left(d_n,\, k(\vartheta)\right) \coloneqq 
            \text{E}_{\vartheta} \left[(d_n(X_1,\, \ldots,\, X_{n}) - k(\vartheta))^2\right]
        .\end{equation}
        Una notazione alternativa è la seguente: \[
            \scriptstyle \text{MSE} \textstyle_{\vartheta}(D_n) \coloneqq 
            \text{E}_{\vartheta}\left[(D_n - k(\vartheta))^2\right] = 
            r_{\vartheta}\left(d_n,\, k(\vartheta)\right)
        .\] 
    \end{defn}
    \begin{obsv}
        Non è possibile trovare uno stimatore che minimizzi l'errore quadratico medio per ogni $\vartheta$, 
        tuttavia possiamo trovarne uno che minimizzi l'errore quadratico medio per una 
        \underline{classe ristretta} di stimatori.
    \end{obsv}
        \begin{defn}[Distorsione]
            Sia $D_n = d_n(X_1,\, \ldots,\, X_{n})$ uno stimatore di $k(\vartheta)$ che ammette media; 
            chiamiamo la \emph{distorsione} di $D_n$ la funzione di $\vartheta$ definita come segue:
            \begin{equation}\label{eq:Distorsione}
                b_{\vartheta}(d_n) \coloneqq 
                \text{E}_{\vartheta}\left[d_n(X_1,\, \ldots,\, X_{n}) - k(\vartheta)\right]
            .\end{equation}
            Una notazione alternativa è la seguente: \[
                \text{bias}_{\vartheta}(D_n) \coloneqq 
                \text{E}_{\vartheta}\left[d_n(\vec{X})\right] - k(\vartheta)
            .\]
        \end{defn}
    \section{Stimatore non distorto}
        \begin{defn}
            Uno stimatore $D_n = d_n(X_1,\, \ldots,\, X_{n})$ si dice \emph{non distorto} oppure 
            \emph{corretto} per $k(\vartheta)$ se vale: \[
                \forall \vartheta \,:\, b_{\vartheta}(d_n) = 0
            .\]
        \end{defn}
        \begin{prty}
            Se $D_n = d_n(X_1,\, \ldots,\, X_{n})$ è uno stimatore di $k(\vartheta)$ basato sul campione 
             $X_1,\, \ldots,\, X_{n}$ che ammette momento secondo finito, allora abbiamo: \[
                 r_{\vartheta}(d_n,\, k(\vartheta)) = 
                 \text{Var}_{\vartheta}[d_n(X_1,\, \ldots,\, X_{n})] + (b_{\vartheta}(d_n))^2
             ;\] allora se $D_n$ è uno stimatore non distorto per $k(\vartheta)$ otteniamo: \[
                b_{\vartheta}(d_n) = 0 \implies r_{\vartheta}(d_n,\, k(\vartheta)) = 
                \text{Var}_{\vartheta}[d_n(\vec{X})]
             ,\] cioè l'errore quadratico medio coincide con la varianza dello stimatore.
        \end{prty}
        \begin{proof}
            Tenendo a mente che $D_n = d_n(X_1,\, \ldots,\, X_{n}) = d_n(\vec{X})$, possiamo scrivere:
            \begin{align*}
                r_{\vartheta}(d_n,\, k(\vartheta)) &= \text{E}_{\vartheta}[D_n -k(\vartheta)^2] \\
                &= \text{E}_{\vartheta}\big[(D_n - \text{E}_{\vartheta}[D_n] + \underset{b_{\vartheta}(d_n)}{\underbrace{\text{E}_{\vartheta}[D_n] - k(\vartheta)}})^2\big] \\
                &= \text{E}_{\vartheta}\big[(D_n - \text{E}_{\vartheta}[D_n] + b_{\vartheta}(d_n))^2\big] \\
                &= \text{E}_{\vartheta}\big[(D_n - \text{E}_{\vartheta}[D_n])^2\big] +
                \big(b_{\vartheta}(d_n)\big)^2 +
                \underset{=\, 0}{\underbrace{2\,b_{\vartheta}(d_n) \cdot \text{E}_{\vartheta}\big[d_n(\vec{X}) - \text{E}_{\vartheta}[d_n(\vec{X})]\big]}} \\
                &= \text{Var}_{\vartheta}\big[d_n(\vec{X})\big] + \big(b_{\vartheta}(d_n)\big)^2
            .\qedhere\end{align*}
        \end{proof}
    \section{Proprietà asintotiche degli stimatori}
        \subsection{Non distorsione}
            \begin{defn}
                Sia $X_1,\, \ldots,\, X_{n}$ una successione di variabili aleatorie indipendenti 
                identicamente distribuite, con densità $F_{\vartheta}$ dipendente da uno o 
                più parametri incogniti $\vartheta$; sia $k(\vartheta)$ una caratteristica 
                della popolazione e $(D_n)_n = (d_n(X_1,\, \ldots,\, X_{n}))_n$ una successione 
                di stimatori  della caratteristica $k(\vartheta)$ che ammettono momento secondo finito;
                allora diremo che la successione di stimatori $(D_n)_n$ è
                \emph{asintoticamente non distorta} per $k(\vartheta)$ se vale: \[
                    \forall \vartheta \,:\, \lim_{n \to \infty} \big[ b_{\vartheta}(d_n) =
                    \text{E}_{\vartheta}[d_n(X_1,\, \ldots,\, X_{n})] - k(\vartheta) \big] \rightarrow 0
                .\]
            \end{defn}
        \subsection{Consistenza debole}
            \begin{defn}
                Una successione di stimatori $(D_n)_n = (d_n(X_1,\, \ldots,\, X_{n}))_n$ è detta \emph{debolmente consistente} per $k(\vartheta)$ se vale: \[
                    \forall \varepsilon > 0 \,:\, \lim_{n \to \infty} 
                    P_{\vartheta}(|D_n - k(\vartheta)| > \varepsilon) \rightarrow 0
                .\] 
            \end{defn}
        \subsection{Consistenza in media quadratica}
            \begin{defn}
                Una successione di stimatori $(D_n)_n ) (d_n(X_1,\, \ldots,\, X_{n}))_n$ è detta 
                \emph{consistente in media quadratica} per $k(\vartheta)$ se vale: \[
                    \forall \vartheta \,:\, \lim_{n \to \infty} \left[r_{\vartheta}(d_n,\, k(\vartheta))
                    = \text{E}_{\vartheta}[(D_n - k(\vartheta))^2]\right] \rightarrow 0
                .\] 
            \end{defn}
            \begin{prty}
                Data una successione di campioni $X_1,\, X_2,\, \ldots$ e una successione di stimatori 
                di $k(\vartheta)$ $(D_n)_n = (d_n(X_1,\, \ldots,\, X_{n}))_n$ allora si ha:
                \begin{enumerate}
                    \item $b_{\vartheta}(d_n) \rightarrow 0 \land \text{Var}_{\vartheta}[D_n] \rightarrow 0 
                        \iff r_{\vartheta}(d_n,\, k(\vartheta)) = \text{Var}_{\vartheta}[D_n] + 
                        (b_{\vartheta}(d_n))^2 \rightarrow 0$;
                    \item se  $(D_n)_n$ è consistente in media quadratica, allora è consistente.
                \end{enumerate}
            \end{prty}
        \subsection{Normalità asintotica}
            \begin{defn}
                Una successione di stimatori $(D_n)_n ) (d_n(X_1,\, \ldots,\, X_{n}))_n$ è detta 
                \emph{asintoticamente normale} per $k(\vartheta)$ se essa è consistente e 
                la distribuzione asintotica degli stimatori e normale.
            \end{defn}
    \section{Proprietà della media campionaria}
    \begin{note}
        Abbiamo già introdotto la media campionaria nella Sezione~\ref{sec:Media_campionaria}, tuttavia 
        ora approfondiamo le sue caratteristiche dal punto di vista del modello statistico.
    \end{note}
        \begin{defn}
            Sia $X_1,\, \ldots,\, X_{n}$ un campione aleatorio estratto da $F$; chiamiamo 
            \emph{media campionaria} la statistica definita come:
            \begin{align}\label{eq:Media_campionaria_statistica}
                \overline{X}_n \coloneqq \frac{X_1,\, \ldots,\, X_{n}}{n} & & \forall n \in \mathbb{N},\, n > 0
            .\end{align}
        \end{defn}
        \begin{prty}
            Indicando con $\mu,\, \sigma^2$ la media e la varianza di una popolazione di cui $\overline{X}_n$ 
            sia la statistica per un campione $X_1,\, \ldots,\, X_{n}$, allora possiamo affermare che:
            \begin{enumerate}
                \item $\text{E}(\overline{X}_n) = \mu$;
                \item $\text{Var}(\overline{X}_n) = \sigma^2 /n$;
                \item $\forall \varepsilon > 0 \,:\, P\left(|\overline{X}_n - \mu| > \varepsilon\right) 
                    \rightarrow 0$;
                \item qualunque sia la distribuzione $F$ comune al campione aleatorio si ha \[
                    \lim_{n \to \infty} \overline{X}_n \simeq \mathcal{N}(\mu,\, \sigma^2 /n)
                .\]
            \end{enumerate}
        \end{prty}
        \begin{proof}
            \hfill
            \begin{enumerate}
                \item $\text{E}(\overline{X}_n) = \text{E}\left(\frac{X_1 + \ldots + X_{n}}{n}\right) = 
                    \frac{\text{E}(X_1) + \ldots + \text{E}(X_n)}{n} = \frac{n\cdot\mu}{n} = \mu$.
                \item $\text{Var}(\overline{X}_n) = \text{Var}\left(\frac{X_1 + \ldots + X_{n}}{n}\right) =
                    \frac{\text{Var}(X_1) + \ldots + \text{Var}(X_n)}{n^2} = \frac{n\cdot\sigma^2}{n^2} =
                    \sigma^2 /n$.
                \item Otteniamo il risultato dall'applicazione del Teorema~\ref{thm:Legge_grandi_numeri_forte}.
                \item Dal Teorema~\ref{thm:Teorema_limite_centrale} possiamo scrivere, 
                    per $n\rightarrow +\infty$: \[
                    P\left(\frac{(\overline{X}_n - \mu)\cdot \sqrt{n}}{\sigma} \leq x\right) \rightarrow
                    \Phi(x) = \frac{1}{\sqrt{2\pi}} \cdot \int_{-\infty}^{x} e^{-u^2 /2}\, du \sim 
                    \mathcal{N}(\mu,\, \sigma^2 /n)
                .\]
            \end{enumerate}
        \end{proof}
    \section{Varianza campionaria}
        \begin{defn}
            Sia $X_1,\, \ldots,\, X_{n}$ un campione estratto da $F$; allora chiamiamo 
            \emph{varianza campionaria} la statistica definita come:
            \begin{align}\label{eq:Varianza_campionaria}
                S_n^2 \coloneqq \frac{1}{n-1}\cdot \sum_{i=1}^{n} (X_i - \overline{X}_n)^2 
                & & \forall n \in \mathbb{N},\, n > 1
            .\end{align}
            Inoltre, chiamiamo \emph{deviazione standard campionaria} la statistica: \[
                S_n \coloneqq  \sqrt{S_n^2}
            .\]
        \end{defn}
        \begin{prty}
            Indicando con $\mu,\, \sigma^2$ la media e la varianza di una popolazione di cui $S_n^2$ 
            sia la statistica per un campione $X_1,\, \ldots,\, X_{n}$, allora possiamo affermare che:
            \begin{enumerate}
                \item $\text{E}(S_n^2) = \sigma^2$;
                \item $\text{Var}(S_n^2) = \frac{1}{n}\cdot[\mu_4 - \frac{n-3}{n-1}\cdot\sigma^4]$, dove
                    $\mu_4 = \text{E}((X_1-\mu)^4)$ e $\sigma^4 = (\sigma^2)^2$.
            \end{enumerate}
        \end{prty}
        \begin{proof}
            \hfill
            \begin{enumerate}
                \item Riscriviamo la definizione \eqref{eq:Varianza_campionaria} nel modo seguente: \[
                        S_n^2 = \frac{1}{n-1} \cdot \sum_{i=1}^{n} \left[X_i^2 - n\, \overline{X}^2\right] 
                        \implies (n-1)\cdot S_n^2 = \sum_{i=1}^{n} \left[X_i^2 - n\, \overline{X}^2\right]
                    .\] Prendiamo il valore atteso di entrambi i membri della precedente equazione, e 
                    ricordiamo che il momento secondo di una variabile aleatoria $W$ qualunque si ottiene 
                    come $m_W(2) \coloneqq \text{E}(W^2) = \text{Var}(W) + \text{E}(W)^2$:
                    \begin{align*}
                        (n-1) \cdot \text{E}(S_n^2) &= \text{E}\left(\sum_{i=1}^{n} X_i^2\right) - 
                        \text{E}(n\, \overline{X}^2) 
                        = n\cdot \underset{m_{X_1}(2)}{\underbrace{\text{E}(X_1^2)}} - n\cdot 
                        \underset{m_{\overline{X}}(2)}{\underbrace{\text{E}(\overline{X}^2)}} \\
                        &= n\cdot \left[\text{Var}(X_1) + \text{E}(X_1)^2\right] -
                        n\cdot \left[\text{Var}(\overline{X}) + \text{E}(\overline{X})^2\right] \\
                        &= n\, \sigma^2 +\bcancel{n\, \mu^2} - \cancel{n}\,\frac{\sigma^2}{\cancel{n}} -\bcancel{n\, \mu^2} \\
                        &= (n-1)\cdot \sigma^2
                    .\end{align*}
                    Allora dalla precedente relazione ricaviamo: \[
                    \text{E}(S_n^2) = \sigma^2
                    .\]
                \item Dimostriamo questa equazione in modo analogo alla precedente. \qedhere
            \end{enumerate}
        \end{proof}
    \section{Distribuzioni delle statistiche di popolazioni normali}
        \begin{thm}\label{thm:Distribuzione_congiunta_statistiche_normale}
            Sia dato un campione $X_1,\, \ldots,\, X_{n}$ estratto da una popolazione normale 
            ($X_i \sim \mathcal{N}(\mu,\, \sigma^2)$), allora $\overline{X}_n$ e $S_n^2$ sono 
            variabili aleatorie \underline{indipendenti} tali che:
            \begin{align}\label{eq:Distribuzione_congiunta_statistiche_normale}
                \frac{\overline{X}_n -\mu}{\sigma /s\sqrt{n}} \sim \mathcal{N}(0,\,1)
                & & (n-1)\cdot \frac{S_n^2}{\sigma^2} \sim \chi^2(n-1)
            .\end{align}
        \end{thm}
        \begin{prty}
            Se $X_1,\, \ldots,\, X_{n}$ è un campione estratto da una popolazione normale di media $\mu$ e 
            varianza $\sigma^2$ allora si ha: \[
                \frac{\overline{X}_n - \mu}{S_n /\sqrt{n}} \sim t(n-1)
            ,\] dove $S_n$ è la deviazione standard campionaria, e $t(n-1)$ è la distribuzione 
            \emph{t} con $n-1$ gradi di libertà.
        \end{prty}
        \begin{proof}
            Ricordiamo la definizione della distribuzione \emph{t}, considerando le variabili 
            aleatorie indipendenti $Z \sim \mathcal{N}(0,\,1)$ e $C_n \sim \chi^2(n)$; allora la 
            \emph{t} di Student è la distribuzione del rapporto: \[
                \frac{Z}{\sqrt{C_n /n}} \sim t(n)
            ;\] usando il Teorema~\ref{thm:Distribuzione_congiunta_statistiche_normale} otteniamo che: \[
                \frac{\overline{X}_n - \mu}{\sigma /\sqrt{n}} \cdot \sqrt{\frac{\sigma^2}{S_n^2} 
                \cdot \frac{n-1}{n-1}} = \frac{\overline{X}_n - \mu}{S_n /\sqrt{n}}
                \sim t(n-1)
            .\qedhere\]
        \end{proof}
        \begin{prty}
            Sia $X_1,\, \ldots,\, X_{n}$ un campione estratto da una popolazione normale di media $\mu_X$ e 
            varianza $\sigma^2_X$; consideriamo anche il campione $Y_1,\, \ldots,\, Y_{m}$ indipendente 
            dal precedente, estratto da una popolazione normale di media $\mu_Y$ e varianza $\sigma^2_Y$.

            Se indichiamo con $S_{n,\,X}^2$ e $S_{m,\,Y}^2$ le varianze delle popolazioni, come:
            \begin{align*}
                S_{n,\,X}^2 = \frac{1}{n-1}\cdot \sum_{i=1}^{n} (X_i - \overline{X}_n)^2
                & & S_{m,\,Y}^2 = \frac{1}{m-1}\cdot \sum_{i=1}^{n} (Y_i - \overline{Y}_m)^2
            ,\end{align*}
            allora possiamo affermare che: \[
            \sigma_X^2 = \sigma_Y^2 \implies \frac{S_{n,\,X}^2}{S_{m,\,Y}^2} 
            \sim \mathcal{F}(n-1,\, m-1)
            .\]
        \end{prty}
        \begin{proof}
            Ricordiamo la definizione della distribuzione \emph{F}, considerando le variabili aleatorie 
            indipendenti $C_n \sim \chi^2(n)$ e $C_m \sim \chi^2(m)$; allora la \emph{F} di Fisher 
            è la distribuzione del rapporto: \[
                \left.\frac{C_n}{n} \middle/ \frac{C_m}{m}\right. \sim \mathcal{F}(n,\,m)
            ;\] usando il Teorema~\ref{thm:Distribuzione_congiunta_statistiche_normale} possiamo scrivere:
            \begin{align*}
                (n-1)\cdot \frac{S_{n,\,X}^2}{\sigma_X^2} \sim \chi^2(n-1)
                & & (m-1)\cdot \frac{S_{m,\,Y}^2}{\sigma_Y^2} \sim \chi^2(m-1)
            ;\end{align*}
            sapendo che si tratta di variabili aleatorie indipendenti, otteniamo: \[
                \left. \frac{(n-1)\, S_{n,\,X}^2}{(n-1)\, \sigma_X^2} \middle/ 
                \frac{(m-1)\, S_{m,\,Y}^2}{(m-1)\, \sigma_Y^2} \right. =
                \frac{S_{n,\,X}^2}{S_{m,\,Y}^2} \sim \mathcal{F}(n-1,\, m-1)
            .\qedhere\]
        \end{proof}
    \section{Metodo dei momenti}
        \begin{defn}
            Sia $X_1,\, \ldots,\, X_{n}$ un campione aleatorio estratto da una densità $F_{\vec{\vartheta}}(x)$ 
            (continua o discreta), con $\vec{\vartheta}$ vettore di parametri incogniti.

            Supponiamo che esistano finiti i primi $k$ momenti della densità $F_{\vec{\vartheta}}(x)$, e 
            indichiamo con $m_F(j)$ il momento $j$\nbdash esimo di $F_{\vec{\vartheta}}(x)$; esso 
            sarà una funzione del parametro $\vec{\vartheta}$: \[
                m_F(j) = \text{E}_{\vartheta}[X_1^{j}] = \begin{cases}
                    \int_{-\infty}^{\infty} x^j\cdot F_{\vec{\vartheta}}(x)\, dx & \text{se $X_1$ è assolutamente continua;} \\
                    \sum_{h} x_n^j\cdot F_{\vec{\vartheta}}(x_h) & \text{se $X_1$ è discreta.}
                \end{cases}
            \] Osserviamo che $m_F(j)$ è una caratteristica della popolazione, poiché essa è la media 
            in corrispondenza di un valore di $\vec{\vartheta}$; possiamo indicarlo con: \[
                \forall j \in [1,\, \ldots,\, k] \,:\, m_F(j) = m_j(\vec{\vartheta})
            .\] Eguagliamo i primi $k$ momenti campionari ai corrispondenti $k$ momenti della popolazione:
            \begin{equation*}
                \begin{cases}
                    \frac{1}{n} \cdot \sum_{i=1}^{n} X_i = m_1(\vartheta_1,\, \ldots,\, \vartheta_k); \\
                    \frac{1}{n} \cdot \sum_{i=1}^{n} X_i^2 = m_2(\vartheta_1,\, \ldots,\, \vartheta_k); \\
                    \hspace{5.5em}\vdots \\
                    \frac{1}{n} \cdot \sum_{i=1}^{n} X_i^k = m_k(\vartheta_1,\, \ldots,\, \vartheta_k).
                \end{cases}
            \end{equation*}
            Otteniamo un sistema di $k$ equazioni nelle incognite $\vartheta_1,\, \ldots,\, \vartheta_k$. Se 
            supponiamo che questo sistema abbia una soluzione indicata come 
            $\vec{\Theta} = (\hat{\Theta}_1,\, \ldots,\, \hat{\Theta}_k)$, dove ciascun $\hat{\Theta}_i$ è una 
            funzione di $X_1,\, \ldots,\, X_n$ (per opportune funzioni $d_i(\vec{X})$):
            \begin{equation*}
                \begin{cases}
                    \hat{\Theta}_1 = d_1(X_1,\, \ldots,\, X_{n}); \\
                    \hat{\Theta}_2 = d_2(X_1,\, \ldots,\, X_{n}); \\
                    \hspace{1.8em}\vdots \\
                    \hat{\Theta}_k = d_k(X_1,\, \ldots,\, X_{n}).
                \end{cases}
            \end{equation*}
            Lo stimatore $\vec{\Theta}$ che soddisfa le proprietà enunciate è chiamato 
            \emph{stimatore del metodo dei momenti}, mentre il valore: \[
                (\hat{\vartheta}_1 = d_1(\vec{x}),\, \ldots,\, \hat{\vartheta}_k = d_k(\vec{x}))
            ,\] in corrispondenza della realizzazione $X_1 = x_1,\, \ldots,\, X_n = x_n$ è chiamato 
            \emph{stima dei momenti} di $\vec{\vartheta}$.
        \end{defn}
    \section{Massima verosimiglianza}
        \begin{defn}[Verosimiglianza]
            Sia $F_\vartheta$ una densità (discreta o continua) dipendente da uno o più parametri incogniti 
            $\vec{\vartheta}$; se $X_1,\, \ldots,\, X_{n}$ è un campione estratto dalla densità $F_\vartheta$, 
            allora la densità del campione si ottiene come: \[
                \forall \vec{x} = (x_1,\, \ldots,\, x_{n}) \,:\, 
                \mathcal{L}(\vartheta,\, \vec{x}) \coloneqq \prod_{i=1}^{n} F_{\vartheta}(x_i)
            .\] Supponiamo di osservare la realizzazione $X_1 = \tilde{x}_1,\, \ldots,\, X_n = \tilde{x}_n$ e 
            consideriamo la funzione di $\vartheta$ detta \emph{verosimiglianza del campione}, definita come:
            \begin{align}\label{eq:Verosimiglianza_campione}
                \mathcal{L}(\vartheta,\, \tilde{\vec{x}}) = \prod_{i=1}^{n} F_\vartheta(\tilde{x}_i)
                & & \text{per } \tilde{\vec{x}} = (\tilde{x}_1,\, \ldots,\, \tilde{x}_n)
            .\end{align}
        \end{defn}
        \begin{obsv}
            Se il campione $X_1,\, \ldots,\, X_{n}$ è costituito da variabili aleatorie discrete, allora 
            la verosimiglianza \eqref{eq:Verosimiglianza_campione} del campione vale:
            \begin{align*}
                \mathcal{L}(\vartheta,\, \tilde{\vec{x}}) &= \prod_{i=1}^{n} F_\vartheta(\tilde{x}_i) =
                \prod_{i=1}^{n} P_\vartheta(X_i = \tilde{x}_i) \\
                &= P_\vartheta(X_1=\tilde{x}_i \cap \dotsm \cap X_n=\tilde{x}_n)
            .\end{align*}
            Se il campione $X_1,\, \ldots,\, X_{n}$ è costituito da variabili aleatorie continue, allora 
            la verosimiglianza \eqref{eq:Verosimiglianza_campione} del campione può essere interpretata 
            come la probabilità di osservare la realizzazione 
            $\tilde{\vec{x}} = (\tilde{x}_1,\, \ldots,\, \tilde{x}_{n})$ per ogni valore fissato di $\vartheta$.
        \end{obsv}
        \begin{defn}[MLE]
            Chiamiamo \emph{stima di massima verosimiglianza} quel valore $\hat{\vartheta}_n$ 
            che massimizza la verosimiglianza \eqref{eq:Verosimiglianza_campione} per ogni realizzazione 
            del campione $\vec{x} = (x_1,\, \ldots,\, x_{n})$, e la indichiamo con:
            \begin{equation}\label{eq:Stima_massima_verosimiglianza}
                \hat{\vartheta}_n \coloneqq \max_{\substack{\vartheta}} \big[\mathcal{L}(\vartheta,\, \vec{x})\big] =
                \max_{\substack{\vartheta}} \left[\,\prod_{i=1}^{n} F_\vartheta(x_i)\right]
            .\end{equation}
            Il corrispondente stimatore è detto stimatore di massima verosimiglianza o \emph{MLE}, e si 
            indica come:
            \begin{align}\label{eq:MLE}
                \hat{\Theta}_n = d_n(X_1,\, \ldots,\, X_{n})
                & & \text{per } \hat{\vartheta}_n = d_n(x_1,\, \ldots,\, x_{n})
            .\end{align}
        \end{defn}
        \begin{obsv}[Log-verosimiglianza]
            Dal punto di vista operativo, sfruttando il fatto che $\mathcal{L}(\vartheta,\,\vec{x})$ e 
            $\ln(\mathcal{L}(\vartheta,\,\vec{x}))$ assumono massimo nello stesso valore, 
            otteniamo la stima massimizzata della verosimiglianza cercando il massimo della funzione 
            \emph{log\nbdash verosimiglianza}: \[
                \ln \big(\mathcal{L}(\vartheta,\, \vec{x})\big)
            .\]
        \end{obsv}
        \begin{prty}[Invarianza degli MLE]
            Supponiamo di voler stimare una caratteristica $k(\vartheta)$; se 
            $\hat{\Theta} = d_n(X_1,\, \ldots,\, X_{n})$ è l'MLE di $\vartheta$ basato sul campione 
            $X_1,\, \ldots,\, X_{n}$ estratto da $F_\vartheta$, allora l'MLE di $k(\vartheta)$ risulta: \[
                \forall k(\circ) \,:\, \hat{K}_n \coloneqq k(\hat{\Theta}_n) = 
                k\big(d_n(X_1,\, \ldots,\, X_{n})\big)
            ,\] dove $k(\circ)$ è una funzione qualunque.
        \end{prty}
        \begin{thm}\label{thm:Successione_MLE}
            Se la densità $F_\vartheta$ soddisfa ``opportune condizioni di regolarità'' e la successione 
            degli MLE di $\vartheta$ è indicata come:
            \begin{align*}
                \hat{\Theta}_n = d_n(X_1,\, \ldots,\, X_{n}) & & \text{per } n \in \mathbb{N},\, n>0
            ,\end{align*}
            e $k(\vartheta)$ è una funzione differenziabile di $\vartheta$, allora possiamo affermare le 
            seguenti proprietà sulla successione $(k(\hat{\Theta}_n))_n$ degli MLE di $k(\vartheta)$:
            \emph{
            \begin{enumerate}
                \item asintoticamente non distorta: \[
                        \lim_{n \to \infty} \text{E}_{\vartheta}[k(\hat{\Theta}_n)] = k(\vartheta)
                ;\]
                \item consistente in media quadratica: \[
                        \lim_{n \to \infty} \text{E}_{\vartheta}\left[\big(k(\hat{\Theta}_n) - k(\vartheta)\big)^2\right] = 0
                ;\]
            \item asintoticamente normale, con media $k(\vartheta)$ e varianza: \[
                    \frac{\sigma^2(\vartheta)}{n} = \frac{(k^{\prime}(\vartheta))^2}
                    {n\, \text{E}_{\vartheta}\left[\left(\frac{\partial}{\partial \vartheta} \cdot 
                    \ln\big(F_\vartheta (X_1)\big)\right)^2\right]}
            .\] 
            \end{enumerate}
            }
        \end{thm}
        \begin{note}
            Il Teorema~\ref{thm:Successione_MLE} non si applica al caso di una popolazione con densità 
            che hanno insieme di definizione dipendente dal parametro $\vartheta$: questo infatti viola le 
            ``opportune condizioni di regolarità'' accennate nel teorema.
        \end{note}
    \section{Intervalli di confidenza}
    \section{Quantità pivotale}
    \section{intervalli di confidenza}
        \subsection{Intervalli di confidenza per media e varianza di popolazioni normali}
        \subsection{Intervallo confidenza per media di esponenziale}
        \subsection{Intervallo confidenza per differenza di medie popolazioni normali}
        \subsection{Intervallo confidenza per rapporto di varianze popolazioni normali}
        \subsection{Intervalli di confidenza asintotici}
        \subsection{Intervalli di confidenza per proporzione}
        

    %! TEX root = main.tex
% Capitolo 6

\chapter{Verifica d'ipotesi}
    \section{Introduzione ai test}
    \section{Ipotesi statistiche}
    \section{Test}
        \subsection{Livello di significatività}
        \subsection{Ampiezza}
        \subsection{Potenza}
        \subsection{\emph{p}\nbdash value}
    \section{Test per media di popolazione normale}
        \subsection{\emph{T}\nbdash test}
        \subsection{\emph{t}\nbdash test}
    \section{Test approssimato su proporzione}
    \section{Confronto media tra popolazioni normali indipendenti}
    \section{Test per campione di coppie di dati normali}
    \section{Test per varianza di popolazione normale}
    \section{Confronto varianza tra popolazioni normali indipendenti}


\end{document}
